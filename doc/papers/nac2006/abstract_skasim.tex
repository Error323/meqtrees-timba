%%--------------------------------------------------------------------------
% Template for abstracts for the NAC 2006.
%
% After \begin{document}:
% Replace the text within the curly braces of the commands
% \abstitle, \absauthor, \absinst, \absinstlist, \abstr, \status, \contact
% with your own text.
%
% Adapted from the NAC 2004 abstract template.
%%--------------------------------------------------------------------------
% Don't make any changes in this section
\documentclass{article}
% formatting commands
\parskip=0.0cm
\parindent=0pt

% abstract commands
\newcommand{\abstitle}[1]{{\bigskip\bigskip \large\bf #1}}
\newcommand{\absauthor}[1]{\par\smallskip{\bf #1}}
\newcommand{\absinst}[1]{$^{#1}$}
\newcommand{\absinstlist}[2]{\par$^{#1}${\small #2}}
\newcommand{\absstatus}[1]{{\par\bf Publication status:} {\em #1}}
\newcommand{\abscontact}[1]{{\par\bf Contact and/or more info:} {\tt #1}}
\newcommand{\abstr}[1]{\par\smallskip#1}

%%--------------------------------------------------------------------------
\begin{document}

% Make your changes here:

\abstitle{Simulating The SKA With MeqTrees}

% put as many authors as needed
\absauthor{O.M. Smirnov\absinst{1}, A.G. Willis\absinst{2}}

% repeat for all affiliations
\absinstlist{1}{NWO/ASTRON, The Netherlands}

\absinstlist{2}{DRAO/NRC, Canada}

\abstr{Future radio telescopes such as LOFAR and SKA present us with a number of
unprecedented challenges. To select a design that will be able to achieve the
SKA requirements, we need extremely elaborate models of the instrument and the
observed sky. This makes detailed SKA simulations a vital part of any design
effort.

The Measurement Equation (ME) provides a succinct mathematical framework in
which an instrument and the observed objects may be described. The MeqTree
module provides a flexible software system for implementing MEs of arbitrary
structure and complexity, and for solving for arbitrary subsets of their
parameters. The poster will examine how the ME and MeqTrees can be applied to
SKA simulations. We will focus on one test case, that of a SKA composed of CLAR
(Canadian Large Adaptive Reflector) dishes, and show detailed simulations of 
instrumental effects and their impact on observations with such a SKA.}

\absstatus{Nederlands Astronomy Conference 2006, submitted}

\abscontact{smirnov@astron.nl}

\end{document}
%%--------------------------------------------------------------------------
