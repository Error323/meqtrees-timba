\chapter{Executing Requests}
  \label{chap:execute}

  The virtual \qq{Node::execute()} method is responsible for processing a
  \Request:
  
  \begin{verbatim}
  virtual int execute (Result::Ref &resref,const Request &req);
  \end{verbatim}
  
  \noindent The node is supplied a \qq{Request} object, and it is expected to return a
  \qq{Result} by attaching it to the counted ref passed in as the first
  argument. The return value is called the {\bf result code}: this
  incorporates the depmask of the result, plus several additional flags such
  as \RES{WAIT} and \RES{FAIL} (see below).

  A parent node will generally call the \qq{execute()} methods of its
  children. In the current single-threaded implementation, the entire tree
  is evaluated via nested \qq{execute()} calls. In a multi-threaded or
  distributed implementation, the parent will probably call stub methods in
  the communication layer, which will in turn call \qq{execute()} on the
  child nodes.

  \highlightbox{The \qq{Node::execute()} method is the fulcrum of the entire
  MeqTree kernel. A solid understanding of how it works is vital for both tree
  design and node development, and also beneficial to advanced users that need
  to deal directly with trees.}

  
\section{Base Node::execute() steps}
   \label{sec:execute}
  
  
  The \qq{Node::execute()} method looks at the request, and calls a number
  of virtual {\em handler} methods for various aspects of processing. Most
  node classes will override one or more of these handler method(s) to
  implement their specific node behaviour. \qq{Node::execute()} also
  provides fundamental node functionality, such as cache management and
  exception handling.\footnote{You may have noted that \qq{execute()} is
  declared virtual. Most node classes will only redefine specific handler
  methods, not \qq{execute()} itself.  The possibility to reimplement
  \qq{execute()} is reserved for the exotic cases. This is not to be
  undertaken lightly, however, as the base \qq{Node::execute()} provides so
  much useful behaviour.}

  The base \qq{Node::execute()} performs a number of processing steps. These
  will now be described in detail, grouped by function.
  

\newcounter{execstep}
\setcounter{execstep}{0}
\newcommand{\ExecStepBox}[2]{\highlightbox{\mbox{#1. } #2}}
\newcommand{\ExecStep}[1]
{\addtocounter{execstep}{1}\ExecStepBox{Step \arabic{execstep}}{#1}}
  
\subsection{Checking the cache}

  \ExecStepBox{Init retunr code}{Set the current return code to 0. It will be
  accumulated in further steps (usually via a bitwise-OR).}

  \ExecStep{Compare the request id to the id of the previous request, if any.
  Sets a local ``new request'' flag that affects further logic below.}

  \ExecStep{If there's a cached \Result, and the request id matches the
  cached rqid/depmask (see \ref{sec:cache} for a discussion), immediately
  return the cached \Result\ and cached result code. On mismatch, clear the
  cache and proceed.}
  
  Note that the caching policy also determines how fail-results are dealt with.
  If the cache contains a fail-result, the node may choose to ignore it and
  attempt to recalculate the result to see if the fail conditions have gone
  away.

  \ExecStep{For new requests only: call the virtual \qq{readyForRequest()}
  handler, and if the return value of that is \qq{false}, immediately
  return the code \RES{WAIT} (result will be empty).}
  
\begin{verbatim}  
  virtual bool readyForRequest (const Request &req);
\end{verbatim}

  \noindent The handler is passed the current \Request, i.e., the \qq{req}
  argument to \qq{execute()} itself. The purpose of this handler is to
  support nodes that block on external events. None as such have been
  implemented, so this is currently just a placeholder. The default handler
  always returns \qq{true}.

  \medskip \ExecStep{For new requests only: if a rider subrecord is present,
  parse it and call the virtual \qq{processCommands()} handler to process
  commands targeted at the node (see details in~\ref{sec:CSR}).}

\subsection{Polling children}
    
  \ExecStep{If node has children, call the virtual \qq{pollChildren()}
    handler to pass the request on to the children and collect their
    results. Bitwise-OR the return value of \qq{pollChildren()} into the current
    return code.}

  \begin{verbatim}    
  virtual int pollChildren (std::vector<Result::Ref> &child_results,
                            Result::Ref &resref,
                            const Request &req);
  \end{verbatim}
    
  \noindent The handler is called with the same \qq{resref} and \qq{req}
  arguments  that were given to \qq{execute()} itself. The
  \qq{child\_results} vector where the child results are returned, it is
  pre-sized to the number of children prior to calling \qq{pollChildren()}.

  The default implementation of \qq{pollChildren()} is appropriate for most
  node classes that pass their requests on to the children unmodified. 
  ``Control'' nodes (e.g. \qq{Sink}, \qq{Solver}, \qq{ModRes},\qq{ReqSeq}) will
  define their own version. This is the default \qq{Node::pollChildren()}
  behaviour:

  \begin{itemize}

  \item Calls \qq{execute()} (with \qq{req}) on all the child nodes, collects
  their \Result{}s (by ref) into the \qq{child\_results} vector, and
  accumulates a return code as a bitwise-OR of the children's \qq{execute()}
  return values. Note that the refs to child \Result{}s are expected to
  be read-only. 

  \item If the accumulated return code has the \RES{FAIL} bit set, its is
  assumed that at least one of the children has returned a fail-result
  (see~\ref{sec:fail}). In this case, \qq{pollChildren()} creates a new
  fail-\Result\ object, attaches it to \qq{resref}, and fills it with 
  all the fail-records found in child results.

  \item The accumulated return code is the return value of the handler.

  \end{itemize}
  
  Note that \qq{resref} is passed to \qq{pollChildren()} only as a means of
  reporting possible fails. If the handler returns a code with \RES{FAIL} in
  it, it should attach a fail-result to \qq{resref}. If no \RES{FAIL} is
  reported, the handler should leave \qq{resref} alone (in fact, anything it
  does to it will be simply ignored when no \RES{FAIL} is returned.)

  If the \qq{pollChildren()} return value contains \RES{WAIT} or \RES{FAIL},
  \qq{execute()} returns (see section~\ref{sec:execute-return}). Otherwise,
  it proceeds to the next step:

  \ExecStep{If auto-resampling is enabled (see \ref{sec:resampling}), compare
  the resolutions of the child results, figure out a common resolution
  (\qq{Cells}) to resample them to, and perform the resampling. Throw an
  exception if this is not possible.}
  
  A result \Cells\ may be initialized based on how the resampling went (see
  \ref{sec:resampling}). 

\subsection{Evaluating \qq{cells}}

  \ExecStep{If the request contains a \qq{cells} command (with a \Cells\
  object), call the virtual \qq{getResult()} handler to process the command,
  passing in the vector of child \Result{}s returned by \qq{pollChildren()}.
  The return value of \qq{getResult()}, along with the node's current depmask
  (see~\ref{sec:depmask-node}), is bitwise-ORed into the current return code.}
  
\begin{verbatim}
  virtual int getResult (Result::Ref &resref,
                         const Cells::Ref &cells,
                         const std::vector<Result::Ref> &child_results,
                         const Request &req,bool newreq);
\end{verbatim}
  
  The \qq{resref} and \qq{req} arguments are the same as those passed to
  \qq{execute()}. The \qq{cells} argument is a ref to the result cells, if any
  were initialized during the resampling stage above, or otherwise to the
  \Cells\ in the request (see~\ref{sec:resampling} for details). The
  \qq{child\_results} vector is built up in \qq{pollChildren()}, it will be
  empty if the node has no children. Finally, the \qq{newreq} flag indicates if
  it is a new request, this flag is set in Step~1 above.
  
  The \qq{getResult()} handler is responsible for attaching a \Result\ object
  to \qq{resref}. In most cases, it will create a new object. Note, however,
  that certain nodes may pass on child results transparently (e.g.,
  \qq{ModRes}, \qq{ReqSeq}), they do this by simply copying a ref from the
  \qq{child\_results} vector. 
  
  If \qq{getResult()} returns a \RES{WAIT} code, it is allowed (and expected)
  to leave \qq{resref} unattached. Otherwise, a valid \Result\ must be
  provided! Any errors occurring inside \qq{getResult()} can be reported by
  throwing an exception.

\subsection{Handling exceptions}

  \ExecStepBox{Error handling}{If an exception is thrown at any stage of the
  process, \qq{execute()} will catch it, create an output \Result\ with a
  fail-result describing the exception, and add \RES{FAIL} to the current
  return code.}

  Thus, throwing an exception is the normal way for \qq{processCommands()},
  \qq{getResult()}, or any other handler to indicate a failure. Note that a
  node should remain in a usable state (i.e. should be able to process further
  \Request{}s) after most exceptions; methods that are liable to leave the node
  in a non-usable state should provide their own exception handling code that
  performs the necessary cleanups and re-throws the exceptions. See the
  \qq{setState()} rollback mechanism described in section
  \ref{sec:state-rollback} for one such example.

\subsection{Caching and returning a \Result}
\label{sec:execute-return}
 
  \ExecStepBox{Returning}{Whenever any kind of \Result\ is returned, it is
  stored in the cache according to the current policy, and returned via the
  \qq{resref} argument. The accumulated return code is returned along with the
  result. If the result is new (i.e. not returned from cache at step 2), the
  \RES{UPDATED} flag is added to the return value.}

  Note that if the accumulated return code contains \RES{WAIT}, then no
  \Result\ is expected (\qq{resref} remains unattached). In all other cases, a
  valid \Result\ object should be attached to \qq{resref} (in case of
  exceptions, this will be a fail-result).

\subsection{All Results are read-only!}

  Prior to returning a \Result, \qq{execute()} recursively changes \qq{resref}
  and all other refs found inside the result object to read-only. This implies
  that the caller of \qq{execute()} (i.e. the parent node, or \qq{MeqServer})
  cannot [legally] change the \Result\ contents. This is deliberate -- the node
  can now hold a ref to the \Result\ in the cache, and be assured that no-one
  can [legally] change the contents. 
  
  In most cases, parent nodes will process the \Result{}s of their children as
  read-only, and discard them afterwards (actually, only the parents' refs are
  discarded -- the \Result\ objects themselves persist if still referenced
  somewhere, e.g., in a child's cache). Some nodes may want to modify child
  \Result{}s ``in place''. To do this, they will need to privatize their refs
  for writing first (see \ref{sec:countedrefs}), thus ensuring that a private
  copy is made if the same object is still referenced somewhere. The same
  applies to individual components of the \Result{}s. Essentially, this is a
  robust copy-on-write mechanism that assures that data is duplicated only when
  needed, with very little effort required from the node developer.
  
  Note also that when a \Result\ is moved across to the scripting layer, some
  sort of copying is implicitly performed. Glish does not deal with \Result\
  objects directly, but rather with their Glish representations.

\section{Result codes}
\label{sec:execute-resultcode}

  As described above, the return value of \qq{execute()} is simply the
  accumulated result code. The result code describes certain properties of the
  returned \Result. Part of it is simply the result depmask (see
  \ref{sec:depmask-result}, the other part contains a number of bitflags listed
  below. Note that the property semantics are defined in such a way that, in
  most practical cases, a flagged property in any child result is inherited by
  the parent's result. This allows \qq{execute()} to accumulate the correct
  result code via a simple bitwise-OR. The following additional flags are
  defined:
 
  \begin{description}
  
  \item[\RES{UPDATED}:] result has changed from that of previous \Request. This
    bit is usually cleared when the node returns a result  from the cache, and
    set otherwise.

  \item[\RES{VOLATILE}:] result may change in response to external events,
    even without new requests. {\em (This is not implemented for now, and only meant
    as a placeholder for future developments, such as dynamically growing
    domains, partial integration, etc.)}

  \item[\RES{FAIL}:] result is a fail. Note that this is not the same thing as
    a result containing some mix of valid and failed VellSets; rather, this
    indicates a failure for the whole result overall. \RES{FAIL}s are usually
    generated in an exception handler. Note that the default implementation of
    \qq{pollChildren()} and \qq{execute()} causes fails to cascade down the
    tree.

    When this flag is returned, a \Result\ object is still expected; it should
    contain one or more fails describing the error. Note that depmasks can be
    meaningfully combined with \RES{FAIL}, to indicate that the fail depends on
    something (i.e. may go away if a particular dependency changes). The
    ``smart'' caching policy may make use of this. 

  \item[\RES{WAIT}:] no result available, wait for notification or try later.
    If this flag is raised, then no \Result\ should be returned. Note that
    dependency flags can be meaningfully combined with \RES{WAIT}, since it
    usually possible to indicate the dependencies of a node in advance. No
    current code uses this, however.

  \end{description}
  
  The return value of a node's \qq{getResult()} method should describe any {\bf
  additional} properties introduced by the \qq{getResult()} calculation
  (additional with respect to the node's current depmask -- see
  \ref{sec:depmask-node}). Most function nodes will return zero, indicating
  no additional dependencies.

\section{Commands in request riders}
  \label{sec:rider}
  \label{sec:CSR}
  \label{sec:rider-setstate}

  The optional \qq{rider} sub-record of a \Request\ is used to send commands to
  nodes. The \qq{Solver} node, for example, relies on this feature to send up
  parameter updates during iterative solutions. Commands are specified as sets.
  Each {\bf command set} is a record (\qq{DataRecord} in C++), with the field
  names (i.e. HIIDs) being the commands per se, and the field value being the
  command value a.k.a. arguments. Commands with no arguments are indicated with
  a boolean \qq{true}, a boolean \qq{false} value implies that the command is
  {\bf not} issued.\footnote{This may seem redundant -- why not simply omit a
  command if it is not meant to be issued? Consider though that Glish has no
  built-in facility for {\em removing} record fields. The ``false=no command''
  convention allows one to effectively remove a command from an existing set by
  assigning the \qq{F} value to it.} If the set contains multiple commands,
  they are processed in a specific order determined by the node class. Some
  example commands are:

  \begin{description}
  
  \item[\qq{state}:] changes the state of a node (available for all nodes);
  
  \item[\qq{add\_dep\_mask}:] adds symdep masks (available for all nodes,
  discussed in \ref{sec:symdeps}).

  \item[\qq{update\_values}:] incremental updates of solvable parameters (see
  documentation for the \qq{Solver} and \qq{Parm} nodes).
  
  \end{description}
  
  Since commands are seen as record fields in Glish, and plain HIIDs in C++, we
  will use both forms (\qq{foo\_bar} and \qq{"Foo.Bar"}) interchangably.
  
\subsection{The command handler}

  Commands are processed by a virtual handler method, called from
  \qq{execute()}:

\begin{verbatim}
  virtual void processCommands (const DataRecord &rec,Request::Ref &reqref);
\end{verbatim}

  The \qq{rec} argument is the command set to be processed (see below). The
  \qq{reqref} argument is a ref to the current \Request. Because the
  \qq{processCommands()} handler is called before \qq{pollChildren()}, it can
  modify the request (i.e., add commands to it) before it is passed on to the
  children. \qq{Node} itself, for example, uses this capability when
  accumulating symdep masks (see \ref{sec:symdeps}). The \qq{reqref} should be
  privatized for writing before a \Request\ is modified, because requests are
  normally passed around as read-only.

  Node classes implementing their own commands will need to redefine this
  handler. It is important, however, that the new handler calls the parent
  class's handler first, so that commands implemented in superclasses
  (especially \qq{Node} itself) are handled properly. Unknown commands should
  be ignored -- in fact, handlers are usually implemented to simply look up
  known commands, and ignore the rest.
  
  Any errors arising during command processing may be indicated by throwing
  exceptions (these will be caught by \qq{execute()}). The handler should take
  care to perform any cleanup, and leave the node in a usable state.

  The base handler, \qq{Node::processCommands()}, processes all standard
  \qq{Node}-level commands. These are listed in section
  \ref{sec:commands-Node}.

\subsection{Rider subrecord layout}
  
  The rider record is structured so that it is possible to associate a command
  set with a specific node or a set of nodes. Each node thus checks if the
  request contains any commands for itself. Note that in a very large tree,
  such repeated checks may grow quite expensive. For this reason, nodes that
  need to receive a lot of commands should be assigned to a {\em node group}. 

  A node may be associated with one or more groups. Each group is identified by
  a \qq{HIID}, a node's group assignments are a part of its dynamic state (see
  \ref{sec:state-Node}). All nodes implicitly belong to the \qq{all} group. A
  node's groups are used as a first-level index into the \qq{rider} record. If
  a node belongs to groups \qq{foo} and \qq{bar}, then \qq{Node::execute()}
  will check for the {\em command subrecords}\/ ({\bf CSR}s) \qq{rider.foo},
  \qq{rider.bar}, and \qq{rider.all}, in that order, and process any CSRs that
  it finds.

  Judicious use of node groups practically eliminates the overhead of command
  lookup. The rider itself contains very few fields (if any), so looking up the
  CSR for a group is very fast. If your tree frequently uses commands to
  control a small subset of nodes (e.g., a solve-tree uses commands to update
  solvable parameters), these nodes should be placed in a group of their own,
  and commands should be kept in that group's CSR. All other nodes will only
  check for the \qq{all} CSR -- which is usually not present. Thus, only the
  required subset of nodes will engage in the possibly expensive business of
  CSR parsing and command processing.
  
  The \qq{all} CSR allows for commands to be sent to any node; but because of
  the associated overhead, this should only be used for infrequent operations
  (e.g., reconfiguring a tree -- as opposed to iterative solving).

\subsubsection{CSR layout}
  
  Each CSR contains a number of command sets. These command sets may be
  associated with specific nodes. There are three ways to specify these
  associations (we will use \qq{csr} here to refer to the CSR itself):

  \paragraph{All nodes in group:} The command set contained in the field
  \qq{csr.command\_all} applies to all nodes in the group. For example:

\begin{verbatim}
  - req.rider.foo.command_all
  [ save_polc=T,state=[solvable=F] ]  
\end{verbatim}

  ...will call \qq{processCommands()} on all nodes in group \qq{foo}, with the
  command set \qq{[save\_polc=T,state=[solvable=F]} (meaning save polcs, and set
  to non-solvable).

  \paragraph{Via node index:} Command sets may be associated with a node index.
  This is done via \qq{csr.command\_by\_nodeindex}, which is essentially
  a map from node index to command set. For example:

\begin{verbatim}
  - req.rider.foo.command_by_nodeindex
  [ #19 = [ value=1,save_polc=T ], #41 = [ value=2,save_polc=T ] ]
\end{verbatim}

  ...will cause a \qq{processCommands()} call on nodes 19 and 41, but only if
  these nodes actually belong to group \qq{foo}. (Note that since Glish only
  supports strings for record field names, the \qq{'\#{\sl ddd}'} form is used
  to specify a ``numeric'' node index.)

  \paragraph{Via lists:} The third way is to associate command sets with  lists
  of nodes matched by name or node index. This is done via a list of records
  in \qq{csr.command\_by\_list}. For example:

\begin{verbatim}
  - req.rider.foo.command_by_list
  [ #1 = [ name="RA DEC",nodeindex=[17,32],
           command=[ save_polc=T,state=[solvable=F] ] 
         ],  
    #2 = [ command=[ state=[solvable=F] ] ] 
  ]
\end{verbatim}

  ...will call \qq{processCommands()} with the first command set on nodes `RA',
  `DEC', 17 and 32 (if they belong to group \qq{foo}), and with the second
  command set on all other nodes in group \qq{foo}. To be more specific, the
  \qq{command\_by\_list} field is treated as a list of records. Each record in
  the list must contain a \qq{command} field (the command set itself),  plus an
  optional \qq{name} field (string or list of strings) and/or an optional
  \qq{nodeindex} field (integer or list of integers). \qq{Node::execute()} will
  iterate through the list of records one by one; if the node's name is found
  in \qq{name}\footnote{In the future, pattern matching will be supported as
  well.}, or the node index is found in \qq{nodeindex}, then
  \qq{processCommands()} is called with the contents of \qq{command}. Once a
  match is found, list processing stops. As a special case, if neither
  \qq{name} nor \qq{index} is specified, then the entry is a ``wildcard''
  matching any node. Wilcards are only useful at the end of the list, to catch
  nodes not matched by previous entries.

\subsection{Command evaluation order}

  As you can see, the rider allowss for more than one command set per node. In
  this case, \qq{processCommands()} will be called several times, once for each
  set. It is important then to define the order in which the rider is parsed:

  \begin{itemize}
  
  \item The outer loop is over node groups, in the order in which they are
  specified in the node's state record. The \qq{all} group is checked last.

  \item Within a group's CSR, the processing order is from least specific to
  most specific: \qq{command\_all}, \qq{command\_by\_list},
  \qq{command\_by\_nodeindex}.

  \item Once a command set is passed to a node's \qq{processCommands()} method,
  the order of processing is determined by the node class implementation.
  Generally, a subclass should call its parent's \qq{processCommands()} first,
  so general commands (such as \qq{state}) will be processed before more
  class-specific commands (such as \qq{save\_polc}).

  \end{itemize}
  
\subsection{Standard node commands}
\label{sec:commands-Node}

  The following is a list of standard commands implemented at the \qq{Node}
  level. Commands are listed in the form of Glish record field names. 
  The actual order of processing is the same one as given here:
  
  \begin{description}
  
  \item[resolve\_children:] resolves all children specified by name into actual
  nodes. An exception is thrown if a name does not match any known node. This
  command is normally issued by \qq{MeqServer} at initialization time, in
  responce to a \qq{"Resolve.Children"} request. This command has no value.
  
  \item[state:] calls \qq{setState()} (see \ref{sec:setState}) with the command
   value. This command thus allows for any field of the dynamic state to be
   modified.
   
  \item[clear\_dep\_mask:] clears accumulated symdep masks (see
  \ref{sec:symdeps}). This command has no value.
  
  \item[add\_dep\_mask:] accumulates symdep masks (see \ref{sec:symdeps}). The
  command value is a record of symdep:mask pairs.

  \item[init\_dep\_mask:] if the node is a dependency generator (see
  \ref{sec:symdeps}), this command causes it to add its generated symdep
  masks to the request, as an \qq{add\_dep\_mask} command placed into
  the \qq{command\_all} field of the CSR for the configured gendep group.
  
  \end{description}

\subsection{Building up command riders in Glish}

  In Glish, the command rider is simply a part of the \qq{request} record, and
  may be created and manipulated just like any other [sub]record. In addition,
  \qq{meq/meqtypes.g} defines some handy shortcuts for adding commands to a
  request:
  
  \begin{verbatim}
  const meq.add_command := function (ref req,group,node,command,value=T);
  \end{verbatim}

  This function modifies the request object passed in via the \qq{req}
  argument. A CSR for \qq{group} will be initialized if required, and a
  \qq{command} with the specified \qq{value} added to the appropriate field of
  the CSR. The \qq{node} argument specifies the target node(s) as follows:

  \begin{itemize}  
  
  \item an empty list (i.e. \qq{[]}) targets command at all nodes. The command
  is added to the \qq{command\_all} set.
  
  \item a single integer is treated as a node index. The command is added
  to \qq{command\_by\_nodeindex} (initializing a command set if necessary).
  
  \item a list of integers is treated as node indices. The command is added to
  \qq{command\_by\_list}, with a \qq{nodeindex} key.
  
  \item a string or a list of strings is treated as node name(s).The command is
  added to \qq{command\_by\_list}, with a \qq{name} key. 
  
  \item an empty string array (\qq{""}) adds a wildcard entry to
  \qq{command\_by\_list}.

  \end{itemize}
  
  A second shortcut is handy for inserting a state update command:

  \begin{verbatim}
  const meq.add_state := function (ref req,group,node,state);
  \end{verbatim}
  
  This will add a \qq{state} command, with the supplied \qq{state} record as its
  value. All other arguments have the same meaning as for \qq{add\_command()}.

\section{Resolution \& resampling}
\label{sec:resampling}

  Resolution \& gridding is a complicated business. Most nodes expect their
  children's results to have the same resolution (i.e. same \qq{Cells}), yet
  performance constraints require our trees to operate at different
  resolutions. \qq{Node} implements a flexible mechanism for automatically
  controlling the resolution of child results. To understand it, it helps to
  take a step back and review some basic requirements:

  \begin{itemize}
  
  \item In the first instance, the grid/resolution is determined by incoming
  data. We'll call this the {\bf full resolution} grid.

  \item It may be prohibitively expensive to evaluate some subtrees (e.g.
  predict) at full resolution. Thus we should support going from full
  resolution to {\bf reduced resolution} ({\bf integration}) and back ({\bf
  upsampling}). 

  \item Parent nodes will not always have sufficient information to determine
  what the best resolution for a child is (i.e. the optimum predict resolution
  may perhaps be only determinable within the predict subtree). Thus, a child
  should be able to  return data at any resolution is deems fit, and let the
  parent deal with it.

  \end{itemize}


\subsection{Treatment of resolution}
  
  To satisfy these requirements, the following behaviour w.r.t. resolution
  is implemented:
  
  \begin{enumerate}
  
  \item The {\bf requested resolution} (or grid) is defined by the \Cells\
  object supplied with the \qq{cells} command of a \Request.

  \item The requested resolution is merely a hint! A node is not obligated to
  honor the grid of the \Cells. It must, however, honor the envelope domain.
  The gridding of the result is indicated by the \Cells\ returned in the
  \Result\ object.
  
  \item Some leaf nodes (e.g. \qq{Const} and \qq{Parm} with no time-frequency
  dependence, when configured to return samplings and not integrations) return
  resolution-free \Result{}s. This is indicated by a missing \Cells\ in the
  \Result. 

  \item Other leaf nodes (e.g. \qq{Parm}, \qq{Freq}, \qq{Time} -- generally,
  nodes meant to evaluate some analytic function over the domain) are easily
  able to evaluate themselves over any given grid. These nodes will always
  return a \Result\ at the requested resolution; the tree designer may rely on
  this behaviour as it is part of the contract for those nodes.

  \item Data-driven leaf nodes (e.g. \qq{Spigot}) completely ignore the
  requested resolution. The gridding of their \Result\ is fully determined by
  the layout of incoming data.

  \item Most non-leaf nodes with multiple children (e.g., most of the
  \qq{Function}-derived nodes) can only operate on child results of the same
  resolution. In the event that different resolutions are returned, any node
  can be configured to {\bf auto-resample} child results to the same
  resolution. This is done by \qq{Node::execute()} prior to calling
  \qq{getResult()}.

  \item Some special nodes (such as \qq{ModRes}) can modify the resolution in
  the request before passing it up.
  
  \end{enumerate}
  
  Note each node's treatment of the requested resolution (i.e.
  honor/ignore/modify) is part of the node class contract, and should be
  clearly documented in the Node Reference. Thus, while parent nodes have no
  control over (or knowledge of) what resoltion their children are going to use
  for their results, the tree designer can always see the global picture, and
  determine which resolution is used where in the tree. By enabling
  auto-resampling at key points in the tree, and possibly adding some
  \qq{ModRes} nodes, the designer can always ensure predictable results.

\subsection{Auto-resampling modes}

  Auto-resampling is enabled via the node state record. The actual resampling
  is done by \qq{Node::execute()}, and it can follow one of several 
  strategies:
  
  \begin{description}
  
  \item[Upsample:] Find the highest resolution among child results, upsample all other
    results to it.

  \item[Downsample:] Find the lowest resolution among child results,  integrate all other
    results to it.
    
  \item[Use Child:] Resample everything to the resolution of a specific child
    (this child is called the {\em resolution driver}\/).

  \item[Use Request:] Resample everything to the resolution of the \Request.

  \item[Fail:] Fail outright (return a fail-result, that is) if resolutions differ.
    This option is mainly useful for debugging.

  \end{description}
  
  The auto-resampling strategy is determined by two fields in the node state
  record (this is dynamic state).
  
  The \qq{resample\_child\_index} enables the ``Use Child'' strategy, if set
  to an ordinal child number ($\ge 1$ in Glish, $\ge 0$ in C++ -- note the \qq{index} suffix and remember automatic
  0-1-base conversion discussed in section \ref{sec:01base}) or a child label
  (see \ref{sec:childlabels}). The specified child becomes the resolution
  driver. If the child returns a resolution-free result (see
  \ref{sec:missingcells}), then a fail is reported.
  

  \begin{description}

  \item[NONE $(=0)$:] do nothing, do not even check the child resolutions. This
  is the default setting initialized by the \qq{Node} class constructor.

  \item[FAIL $(=-2)$:] check resolutions and return a \qq{RES\_FAIL} if they do
  not match. This is the default setting initialized by the \qq{Function} class
  constructor. Since this is somewhat slower that the \qq{NONE} setting, the
  default for optimized builds may possibly be changed to \qq{NONE} in the
  future.

  \item[INTEGRATE $(=-1)$:] find the lowest resolution among child results,
  integrate all other results to this resolution.

  \item[UPSAMPLE $(=1)$:] find the highest resolution among child result,
  upsample all other results to this resolution.

  \item[REQUEST $(=2)$:] resample all child results to the resolution of the
  \Request.

  \end{description}
    
  Auto-resampling is handled at the base \qq{Node} level (in
  \qq{Node::processChildren()}, and thus may be enabled for any individual node
  (although node classes that do their own polling of children -- e.g.,
  \qq{Solver} -- may ignore the flag.) In the rippled solvers tree (Fig. ??),
  nodes which enable auto-resampling are indicated by an "UPS" or "INT" label
  in the node box. 

  \item A few utility nodes may be used to change resolution mid-tree.
  Initially, two such nodes will be provided:

    \begin{description}
    
    \item[\qq{Resampler}:] this node guarantees a \Result\ at the requested
    resolution. A \qq{Resampler} has a single child. It passes the parent's
    \Request\ on to the child, and if the returned \Cells\ are different from
    the requested ones, then it resamples the result to the requested \Cells\
    before returning it to the parent.

    \item[\qq{ModRes}:] this node modifies a \Request's resolution up or down
    by a fixed factor (or to a fixed number of cells along either axis), as as
    determined by its state record (dynamic configuration). This node has a
    single child. The modified \Request\ is passed on to its child, and the
    child's \Result\ is returned.

    \end{description}
    
  In the future, we envision more nodes, such as an adaptive resolution
  reducer, which adaptively selects a minimum required resolution based on the
  results of the child. At the moment, the \qq{ModRes} node is a suitable
  proxy. 

  \item Data access nodes (\qq{Sink} and \qq{Spigot}) are coupled. For each
  snippet of data, the \qq{Sink} will issue a \Request\ with \Cells\
  corresponding to the data layout. The corresponding \qq{Spigot} will then be
  able to return a \Result\ with the same cells.
  
  \item For the time being, we'll only support integral resolutions -- i.e.,
  the full resolution cells must be tilings of the reduced resolution cells.
  This keeps things simple, and avoids interpolation errors. In the future, we
  may support arbitrary regridding.

  \end{enumerate}
  
  It is, of course, up to the tree designer to ensure that resolution is
  changed up and down appropriately within the tree, by strategically
  positioning \qq{Resampler} and \qq{ModRes} nodes, and enabling
  auto-resampling at a few specific nodes. The tree in Fig. ?? provides a good
  example of this.

\chapter{The MeqServer Interface}
\label{chap:meqserver}
\label{sec:meqserver}
\label{sec:meqforest}

  Coming soon.

\chapter{Description of specialized nodes}

\section{Function nodes}
\label{sec:Function}

\section{ReqSeq}

  \qq{ReqSeq} is the {\em request sequencer} node. It is used to sequence the
  execution of requests between subtrees.
  
  Normally, if a node has multiple children, they may recieve \& execute their
  requests in parallel, with no predefined order (note that in a
  single-threaded system, they will effectively execute in sequence, however,
  you cannot design a tree to rely on this behaviour). Some applications
  require that branches of the tree are evaluated in a specific order. For
  example, in the rippled solvers tree (Fig. ??), the Predict--Solver branch
  must be executed first, to determine values for the solvable parameters,
  followed by Predict--Subtract.
  
  The sequencer (\qq{ReqSeq}) node provides an easy way to sequence these
  requests. A sequencer may have any number of children. Upon receiving a
  request from its parent, the sequencer will pass it on to the first child,
  and wait for the first child to return a result. Then it will go on to the
  second child, etc. For its own result, the sequencer returns the result of
  one of the children, as determined by its state record (run-time
  configurable).

\section{Resampler}

\section{ModRes}

\section{MeqParm}

  The \Parm\ node implements a possibly solvable Measurement Equation
  parameter. The parameter is represented by one or more \Polc\ objects.

\subsection{MeqPolc}

  The \Polc\ class implements a 2D polynomial in time and frequency. The Glish
  equivalent is a \qq{meq.polc} record. This record will contain the following
  fields:
  
  \noindent\begin{center}\begin{tabular}{lp{.8\textwidth}}
  \qq{.coeff}  &  a 2D array of polynomial coefficients.\\
  \qq{.freq\_0}  &  \\
  \qq{.freq\_scale}  &  \\
  \qq{.time\_0}  &  \\
  \qq{.time\_scale}  & the scale of the polc (see below)\\
  \qq{.domain}  & (optional) the polc \Domain.\\
  \qq{.weight}  & weight\\
  \qq{.pert}  & perturbation to use for computing derivatives\\
  \qq{.dbid\_index}  & database ID (see below)\\
  \qq{.inf\_domain} &  optional flag: infinite domain (see below)\\
  \qq{.grow\_domain} &  optional flag: growing domain (see below)\\
  \end{tabular}\end{center}
  
  The value of a polc for frequency $f$ and time $t$ is computed as follows:

  \begin{equation}
  p(f,t) = \sum_{i=0}^{N-1}\sum_{j=0}^{M-1} c_{ij}(\frac{f-f_0}{s_f})^i(\frac{t-t_0}{s_t})^j
  \end{equation}
  
  Here, $c_{ij}$ is an $N\times M$ array of coefficients (\qq{coeff}), and
  $f_0,s_f,t_0,s_t$ is the scale of the polc. The scale is only necessary to
  keep numbers small so as to avoid round-off errors.

\subsection{MeqParm state}

  Here's the layout for a \Parm\ state record. All of these attributes are
  dynamic state, that is, they may be changed at any time via \qq{setState()}:
  
  \begin{description}

  \item[\qq{solvable}] (bool, optional) is this \Parm\ solvable or not?
    Default is non-solvable.

  \item[\qq{polcs}] (list of \Polc{}s, optional) the parm's polc list (see below).

  \item[\qq{table\_name}] (string, optional) the name of a MEP table. Default is none.
    If no table is provided, then polcs must be specified via either the
    \qq{polcs} or the \qq{default} field.

  \item[\qq{parm\_name}] (string, optional) the name to use when working with a MEP
    table. Default is to use the node name itself.

  \item[\qq{default}] (\Polc, optional) a default \Polc, to be used as a last
    resort (i.e. if no polcs found in the MEP table).

  \item[\qq{auto\_save}] (bool, optional) If true, any updates to the parm's 
    polcs are immediately saved to the MEP table. If false, updates need to be
    explicitly saved (more on this below). Default is false.

  \end{description}
  
\subsection{Selecting polcs}

  The \qq{Parm::getResult()} method is responsible for evaluating a parm over a
  \Cells\ (specifically, over the domain of the \Cells). To do this, the parm
  needs to find an appropriate polc or polcs for the given domain. Most of the
  tricky logic of the \Parm\ class is dedicated to choosing the right set of
  polcs.  Currently, it will select polc(s) as follows:

  \begin{enumerate}
  
  \item If no MEP table and no default polc has been set in the state record,
    then the parm will blindly re-use the current polc list (i.e. the
    \qq{polcs} field of the state record), or fail if the list is empty.  No
    further error checking is done, and it is up to the user to ensure that the
    parm has been initialized with polcs that are meaningful w.r.t. the domain
    of the request.

  \item Otherwise, the parm will first see if it can re-use the current polc 
    anyway. The current list must contain a single polc, which is tested for
    re-usability as follows:

    \begin{enumerate}
    
    \item If the \qq{inf\_domain} flag is set, the polc has an infinite domain, 
      and can be re-used.
      
    \item If the \qq{grow\_domain} flag is set, and the requested domain is a
      superset of the polc's current domain, then the domain of the polc is
      expanded to match the requested domain, and the polc can be re-used. 

    \item If the requested domain is a subset of the polc's current domain, the
      polc is re-used.

    \end{enumerate}

  \item If no re-use is possible, the current polc list is cleared. The parm
    will then query its MEP table for polcs whose domain overlaps the requested
    domain. These polcs are loaded into the polc list.

  \item If no polcs are found in the MEP table, the table is checked for a
    default polc. Note that default polcs have no domain (generally, they will
    only have a $c_{00}$ coefficient). If a default polc is found, it is placed
    in the list, and its domain is set to the requested domain.

  \item If no default polc is found in the table, or if no table is available,
    then default polc from the state record is copied into the list, and its
    domain is set to the requested domain. Note that the case of no table and
    no default polc is covered by (1), above.

  \end{enumerate}
  
  Once the selection is complete, the parm will have at its disposal a list of
  one or more polcs. 

\subsection{Selecting a single solvable polc}
  
  If the parm is set to solvable, the polc list is then culled to a single polc
  (NB: future versions may support solving for multiple polcs). This polc is
  selected according to the following criteria (in descending order of
  importance):

  \begin{itemize}
  
  \item an exact match of the requested domain to the polc domain;
  
  \item weight (higher is better);
  
  \item database ID (higher, i.e. more recent, is better).
  
  \end{itemize}
  
  Once a single solvable polc is selected, its domain is set equal to the
  requested domain.
  
\subsection{Evaluating the polc list}

  A single polc is evaluated directly over the \Cells\ of the incoming
  \Request. The resulting \VellSet\ object (polc value, plus optional
  derivatives) is then returned to the caller. 
  
  If the parm contains multiple polcs, different evaluation schemes are
  possible:

  \begin{description}

  \item[Tiled polcs.] This is the only scheme implemented by \Parm\ at time of
  writing. For every polc in the list, \Parm\ computes the overlap between
  the polc's domain and the domain of the request. The polc is then evaluated
  over that part of the \Cells\ grid which falls within the overlapping area.
  This step is repeated for every polc. Sections of the grid with no polc
  domain coverage are assigned a zero value.

  Obviously, this scheme is most suitable when the polcs neatly tile
  the request domain. Note that the result is generally discontinuous across
  tile boundaries, which can be quite useful for representing things such as
  phase jumps. On the other  hand, if the request domain is not completely
  covered, or if the polc domains overlap, the results of this scheme are not
  very well defined. 

  \item[Weighted mean.] Each polc is associated with a set of weights, defined
  on the \Cells\ grid. Different weighting schemes may be employed. Presumably,
  grid points within the polc's domain are assigned a higher weight, while
  points outside the domain get a lower weight, further decreasing as we get
  further away from the domain. The value of the parm at each grid point is
  then simply a weighted mean of all the polcs' values.

  The advantage of this scheme is that it gracefully incorporates polcs with
  overlapping domains, while allowing for extrapolation to non-covered areas of
  the request domain. The result is always smooth and continous -- which, on the
  other hand, may not always be what you want. The downside is computational
  expense, as each polc needs to be evaluated at each grid point.

  \item[Reduction to single polc.] This scheme involves fitting a single polc
  (usually of a higher order) to multiple polcs. It can already be implemented
  externally (see \qq{fitpolcs\_wlc.g} for an example), as a ``preprocessing''
  stage of sorts. The scheme works by evaluating the polcs over some set of
  points within the request domain (not necessarily the \Cells\ grid at all),
  combining the results using some sort of weighted mean, then doing a
  least-squares fit of a new polc to the resulting values.

  \end{description}

\subsection{Updating and saving polcs}    

  In the course of a solution, the Solver node updates the values of solvable
  polcs by sending up new values in the request rider. This is done by
  including the \qq{Update.Values} command in the rider. The value of the
  command is expected to be a vector of new coefficients, in the same order in
  which spids were assigned. More specifically:

  \begin{itemize}
  
  \item When a \Parm\ that has been set solvable initializes a new polc for a
  domain, it associates a number of spids with its coefficients. Spids are
  assigned as $256*${\em nodeindex}$+i$, where $i$ is the number of the
  coefficient.
  
  \item These spids are included in the resulting \Vells; as results percolate
  down the tree, spid vectors and corresponding perturbed values from different
  solvable \Parm{}s are merged. The \Vells\ received by the solver contain the
  full set of spids from all the solvable \Parm{}s in its trees.

  \item The Solver computes a set of incremental updates. Note that the Solver
  knows nothing of \Polc{}s or \Parm{}s; instead, it only deals with abstract
  ``atomic parameters'' identified by spid.

  \item The solver inserts the updates into the rider of the next request, as
  a set of \qq{Update.Values} commands. This request is then used to recalculate
  the trees for the next iteration. If this is the last iteration, then a
  request without a \Cells\ is sent up -- this updates the values without
  recalculating anything.
  
  \end{itemize}
  
  Presumably, at some point the new polc values need to be stored into a MEP
  table. There are two ways to accomplish this:

  \begin{description}
  
  \item[Automatically:] if the \qq{auto\_save} flag is set in the \Parm\ state
  record, then each update via \qq{set\_value} is immediately committed to the
  MEP table.

  \item[With explicit command:] whenever the \Parm\ receives a request with a
  \qq{Save.Polcs} command in the rider, it commits all its polcs to the MEP
  table.

  \end{description}
  
  The \qq{dbid\_index} attribute of the \Polc\ is used to keep track of its
  location in the MEP table. When a polc is loaded from the table (whether on
  the C++ side, or via the Glish \qq{meptable()} object), \qq{dbid\_index} is
  set to its DB identifier.\footnote{As long as we use AIPS++ tables, this is
  simply the row number. If and when we employ other storage schemes, dbid  may
  become something different, but is in any case a key into the database.} 
  When a polc is subsequently saved, its dbid is used to locate the correct
  entry in the table. New polcs are created with a dbid of -1; when they are
  subsequently saved, a new entry is allocated in the MEP table, and the polc
  object is updated with the dbid. During normal operation, the user need not
  worry about this since everything happens automatically; some advanced
  scripting may require knowledge of the dbid.


