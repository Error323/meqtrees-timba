%ADASS_PROCEEDINGS_FORM%%%%%%%%%%%%%%%%%%%%%%%%%%%%%%%%%%%%%%%%%%%%%%%
%
% SAMPLE1.TEX -- ADASS XII (2002) ASP Conference Proceedings sample
% paper with minimal markup. Based on the sample from ADASS XI (01).
%
% This is a simple example.  If you want to see a more comprehensive
% sample paper,  take a look at sample2.tex.
%
% Much of the input will be enclosed by braces (i.e., { }).  The
% percent sign, "%", denotes the start of a comment; text after it
% will be ignored by LaTeX.  You might also notice in some of the
% examples below the use of "\ " after a period; this prevents LaTeX
% from interpreting the period as the end of a sentence and putting
% extra space after it.   
% 
% You should check your paper by processing it with LaTeX.  For
% details about how to run LaTeX as well as how to print out the User
% Guide, consult the README file.  
%
% If you do not have access to the LaTeX software or a laser printer
% at your site, you can still prepare your paper following the
% instructions in the User Guide.  In such cases, the editors will
% process the file and make any necessary editorial adjustments.
% 
%%%%%%%%%%%%%%%%%%%%%%%%%%%%%%%%%%%%%%%%%%%%%%%%%%%%%%%%%%%%%%%%%%%%%%%%
% 
\documentclass[11pt,twoside]{article}  % Leave intact
\usepackage{adassconf}

% If you have the old LaTeX 2.09, and not the current LaTeX2e, comment
% out the \documentclass and \usepackage lines above and uncomment
% the following:

%\documentstyle[11pt,twoside,adassconf]{article}

\begin{document}   % Leave intact

%-----------------------------------------------------------------------
%			    Paper ID Code
%-----------------------------------------------------------------------
% Enter the proper paper identification code.  The ID code for your
% paper is the session number associated with your presentation as
% published in the official conference proceedings.  You can
% find this number locating your abstract in the printed proceedings
% that you received at the meeting or on-line at the conference web
% site; the ID code is the letter/number sequence proceeding the title 
% of your presentation.  
%
% This will not appear in your paper; however, it allows different
% papers in the proceedings to cross-reference each other.  Note that
% you should only have one \paperID, and it should not include a
% trailing period.
%

\paperID{P78 }

%-----------------------------------------------------------------------
%		            Paper Title 
%-----------------------------------------------------------------------
% Enter the title of the paper.
%
% EXAMPLE: \title{A Breakthrough in Astronomical Software Development}
%
% If your title is so long as to fill the page header when you print it,
% then please supply a short form as a \titlemark.
%
% EXAMPLE:
%  \title{Rapid Development for Distributed Computing, with Implications
%         for the Virtual Observatory}
%  \titlemark{Rapid Development for Distributed Computing}
%

\title{MeqParm: Parameter Handling in the MeqTree System}
%\titlemark{ }

%-----------------------------------------------------------------------
%		          Authors of Paper
%-----------------------------------------------------------------------
% Enter the authors followed by their affiliations.  The \author and
% \affil commands may appear multiple times as necessary.  List each
% author by giving the first name or initials first followed by the
% last name.  Authors with the same affiliations should grouped
% together. 
%
% Try to limit the front matter to no more than three \author
% commands.  Group authors with the same affiliations.  Too many
% \author commands fills the first page of the paper with little
% actual text.

\author{M. Mevius, O.M. Smirnov, J.E. Noordam}
\affil{ASTRON, P.O. Box 2, 7990AA Dwingeloo, The Netherlands}
%-----------------------------------------------------------------------
%			 Contact Information
%-----------------------------------------------------------------------
% This information will not appear in the paper but will be used by
% the editors in case you need to be contacted concerning your
% submission.  Enter your name as the contact along with your email
% address.

\contact{M. Mevius}
\email{mevius@astron.nl }

%-----------------------------------------------------------------------
%		      Author Index Specification
%-----------------------------------------------------------------------
% Specify how each author name should appear in the author index.  The 
% \paindex{ } should be used to indicate the primary author, and the
% \aindex for all other co-authors.  You MUST use the following
% syntax: 
%
% SYNTAX:  \aindex{LASTNAME, F. M.}
% 
% where F is the first initial and M is the second initial (if
% used).  This guarantees that authors that appear in multiple papers
% will appear only once in the author index.  
%
% EXAMPLE: \paindex{Crabtree, D.}
%          \aindex{Manset, N.}
%          \aindex{Veillet, C.}
%
% NOTE: this information is also used to build the author list that
% appears in the table of contents.  Authors will be listed in the order
% of the \paindex and \aindex commmands.
%

\paindex{Mevius, M.}
\aindex{Smirnov, O.M.}     
\aindex{Noordam, J.E.}     

%-----------------------------------------------------------------------
%                     Author list for page header
%-----------------------------------------------------------------------
% Please supply a list of author last names for the page header. in
% one of these formats:
%
% EXAMPLES:
% \authormark{LASTNAME}
% \authormark{LASTNAME1 \& LASTNAME2}
% \authormark{LASTNAME1, LASTNAME2, ... \& LASTNAMEn}
% \authormark{LASTNAME et al.}
%
% Use the "et al." form in the case of seven or more authors, or if
% the preferred form is too long to fit in the header.

\authormark{Mevius, Smirnov \& Noordam}

%-----------------------------------------------------------------------
%			Subject Index keywords
%-----------------------------------------------------------------------
% Enter up to 6 keywords describing your paper.  These will NOT be
% printed as part of your paper; however, they will be used to
% generate the subject index for the proceedings.  There is no
% standard list; however, you can consult the indices for past ADASS
% proceedings (http://adass.org/adass/proceedings/).

\keywords{astronomy: radio, calibration, selfcal, MeqTrees, interferometry,
polarimetry}

%-----------------------------------------------------------------------
%			       Abstract
%-----------------------------------------------------------------------
% Type abstract in the space below.  Consult the User Guide and Latex
% Information file for a list of supported macros (e.g. for typesetting 
% special symbols). Do not leave a blank line between \begin{abstract} 
% and the start of your text.

\begin{abstract}          % Leave intact
The MeqTree module provides a flexible system to implement arbitrary
Measurement Equations and to solve for (subsets of) its parameters. Within the framework of MeqTree, the MeqParms
are the (solvable) parameters of the Measurement Equation. The
MeqParms represent functions, the coefficients of
which are the parameters that are in fact solved for.
Examples of MeqParms are discussed, as well as methods for parameter bookkeeping and evaluation, 
and tools for online inspection and adjustment.


\end{abstract}

%-----------------------------------------------------------------------
%			      Main Body
%-----------------------------------------------------------------------
% Place the text for the main body of the paper here.  You should use
% the \section command to label the various sections; use of
% \subsection is optional.  Significant words in section titles should
% be capitalized.  Sections and subsections will be numbered
% automatically. 

\section{Introduction}

The MeqTree module provides a flexible tool for solving subsets of
parameters of arbitrary Measurement Equations (M.E.) (Smirnov \&
Noordam, this volume). The module is designed to cope with the
challenges of calibration of the next generation  radio
interferometers. Its design is tree/node based, each node class
representing a small step in the total calculation of the tree, mostly simple
mathematical operations. A node passes requests to its children,
receives their values on the requested domain, performs its operation
 and passes the result to its parent. 

A special class of $leaf$ nodes (a node without
children) is the MeqParm. The MeqParms are the (solvable)
parameters of the M.E. A MeqParm represents a function that can be dependent on any
axis of variability. The parameters can correspond to instrumental
and other effects but can also define a source model. The Local and Global Sky Model (a problem
specific, respectively all sky database) (Smirnov\& Noordam, 2004 and Nijboer, Noordam \&
Yatawatta, this volume ) can be implemented using  MeqParms. This 
offers the possibility to include more sophisticated sources and
time/frequency variability. 

This paper gives a description of the current implementation and
the possibilities of the MeqParm. 

\section{Funklets} 

A parameter node in the MeqTree system represents a function $f(x_0,..,x_n|c_0,..,c_m)$ on a given
domain for $x_i$. The coefficients $c_i$ are the parameters that, in case of a solvable MeqParm, are in
fact solved for. The functions represented by a MeqParm will be
denoted  $funklets$. The default funklet is a two dimensional
polynomial of arbitrary order in frequency and time:
\begin{equation}
f(f,t) = c_{00} + c_{10}\cdot f + c_{01} \cdot t + c_{11} \cdot f\cdot
t + c_{20}\cdot f^2 + ...
\end{equation}
Several predefined funklets are available, but in principle any other (real)
function can be defined with the help of the Aips++ Functional
class, which has the possibility to interpret strings as functions.
Functions of large complexity can be described by a single funklet. A more sophisticated
example is a 'beamshape' model: a two dimensional Gaussian with its coefficients
polynomials  in frequency and time:
\begin{equation}
g(l,m) = exp(-(f_{00}+f_{10}\cdot l+f_{01}\cdot m+f_{11}\cdot l\cdot m +
f_{20}\cdot l^2 + f_{02}\cdot m^2)),
\end{equation}
with each $f_{ij}$ a frequency/time polynomial as defined above. 
The Python based  interface to the MeqTree system eases the definition of such funklets considerably.

%\begin{figure}
%\plotone{P.78_1.eps}\label{gaussian}
%\caption{A two dimensional Gaussian, its coefficients possibly
%frequency/time dependent. This function can be easily
%implemented as a single MeqParm.}
%\end{figure}

\begin{figure}\label{parmtable}
\plotone{P.78_1.eps}
\caption{Schematic representation of a ParmTable, with several
funklets available for different domains.}
\end{figure}


\section{Parameter Evaluation}
All information of the funklet can be stored in and retrieved from
(Aips++) tables (ParmTable). A funklet is valid for a specific
domain and several funklets can belong to a single MeqParm, eg. 
because independent solutions were generated for different
subdomains (Fig. 1).
Different domains can have different types of funklets. Upon receiving
a request for a specific domain the MeqParm always returns a
values. If no suitable funklet is found in the connected ParmTable,
the best solution should be determined from either a default funklet
or via inter-/extrapolation. Inter- and extrapolation routines can be especially important in
calibration runs, where calibration data is  taken before and/or after
the actual measurement.

If there are several funklets available, each one valid on different subdomains
of the larger request domain, the MeqParm is evaluated separately on
the subdomains. 
The user is also able to specify this by
dividing its solvedomain into regular tiles. In this case, a solvable
MeqParm will have an independent solution for every tile.
This offers the user maximum flexibility in deciding the strategy to solve
complex problems on large domains. Since all information is stored, a fit on the several
independently solved subdomains can always be performed afterwards.

In the case of solvable parameters not only the values of the MeqParm
are determined for a requested domain, but also perturbed values for
all coefficients of the funklet(s). These perturbed values are
transported down through the tree, to the point where they are used to
determine the derivatives of the condition equations.

In non-linear problems, many iterations are needed to converge to the
right solution. The number of iterations can be considerably reduced
by smart initialization of the MeqParm. An example is shown
schematically in
Fig. 2, where requests are generated for subsequent
time slots. By fitting the variation of the coefficient values over
time, the next value can predicted. This method can also be used to trace
and discard outliers.
\label{snippetsolver}
\begin{figure}
\plotone{P.78_2.eps}
\caption{An example in which a MeqParm receives requests for
subsequent time slots. The funklet's coefficients for the next time slot
are predicted by a linear fit to the previous values. The information
of the difference between predicted and actual value can be used to
discard outliers.}
\end{figure}



\section{Online Tools}

The Python based GUI ($meqbrowser$) that is part of the MeqTree module,
allows for quick inspection/visualization of any node at any stage of the
solve-process. Within the browser two special tools have been developed for the MeqParm.
One tool provides a graphical display of the contents of the ParmTable,
i.e. the solutions of a certain parameter at different validity
domains. The other, the 
parameter adjustment tool, with which it is possible to change online the
solvability, coefficient values etc., is especially useful in the design and
tests of complex trees.   

\begin{figure}
\plotone{P.78_3.eps}
\caption{Parameter Fiddling Tool. Lists all parameters
under a certain node, allows for adjusting them and also offers the
possibility to reexecute the node, to inspect the effects of the
changes made.}
\end{figure}




%-----------------------------------------------------------------------
%			      References
%-----------------------------------------------------------------------
% List your references below within the reference environment
% (i.e. between the \begin{references} and \end{references} tags).
% Each new reference should begin with a \reference command which sets
% up the proper indentation.  Observe the following order when listing
% bibliographical information for each reference:  author name(s),
% publication year, journal name, volume, and page number for
% articles.  Note that many journal names are available as macros; see
% the User Guide for a listing "macro-ized" journals.   
%
% Note the following are some of the tricks that can be used:
%
%   o  \& is used to format an ampersand symbol (&).
%   o  \'e puts an accent grave over the letter e.  See the User Guide
%      for details on formatting special characters.  
%   o  "\ " after a period prevents LaTeX from interpreting the period 
%      as an end of a sentence.
%   o  \aj is a macro that expands to "Astron. J."  See the User Guide
%      for a full list of journal macros
%   o  \adassvii is a macro that expands to the full title, editor,
%      and publishing information for the ADASS VII conference
%      proceedings.  Such macros are defined for ADASS conferences I
%      through IX.
%   o  When referencing a paper in the current volume, use the
%      \adassix and \paperref macros.  The argument to \paperref is
%      the paper ID code for the paper you are referencing.  See the 
%      note in the "Paper ID Code" section above for details on how to 
%      determine the paper ID code for the paper you reference.  
%
\begin{references}
\reference Smirnov, O.M., Noordam, J.E. \ 2004, \adassxiii
\reference Smirnov, O.M., Noordam, J.E. \ 2006, \adassxv, \paperref{P.173}
\reference Nijboer, R.J., Noordam, J.E. \& Yatawatta, S.\ 2006,
\adassxv, \paperref{P.57}

\end{references}

% Do not place any material after the references section

\end{document}  % Leave intact
