\documentclass[12pt,twoside]{book}

\usepackage{graphicx}
\usepackage{booktabs}
\usepackage{color}
\usepackage{colortbl}

\oddsidemargin=-5mm
\evensidemargin=-5mm
\topmargin=-5mm
\textwidth=170mm
\textheight=240mm

\pagestyle{myheadings}

% \qq{text} used to quote code (in fixed font)
\newcommand{\qq}[1]{{\tt #1}}

% \url{text} inserts a URL
\newcommand{\url}[1]{{\tt #1}}

% commands for some common Meq-terms
\newcommand{\Request}{{\tt Request}}
\newcommand{\RequestId}{{\tt RequestId}}
\newcommand{\Result}{{\tt Result}}
\newcommand{\VellSet}{{\tt VellSet}}
\newcommand{\Cells}{{\tt Cells}}
\newcommand{\Vells}{{\tt Vells}}
\newcommand{\Domain}{{\tt Domain}}
\newcommand{\Node}{{\tt Node}}
\newcommand{\Parm}{{\tt Parm}}
\newcommand{\Polc}{{\tt Polc}}
\newcommand{\RES}[1]{{\tt RES\_#1}}

% math symbols: real numbers, complex numbers, P for parameter space
\newcommand{\RR}{\mathcal{R}}
\newcommand{\CC}{\mathcal{C}}
\newcommand{\PP}{\mathcal{P}}

% define some standard colors
\definecolor{tableheading}{rgb}{0.8,0.85,1}
\definecolor{tabletext}{gray}{0}
\definecolor{tablesubheading}{gray}{.9}
%
\definecolor{highlight}{rgb}{1,1,0}

% this command creates a paragraph with a highlit background, separated
% by extra space from surrounding paragraphs.
\newcommand{\highlightbox}[1]{\smallskip\noindent\colorbox{highlight}{\parbox{\textwidth}
{\noindent #1}}\smallskip}%
% \addvspace{\smallskipamount}}

% \tablesubheading{n}{text} inserts a sub-heading into a table.
% n argument is the number of columns that the sub-heading will span

  \newlength{\taboverhang}
  \setlength{\taboverhang}{\tabcolsep}
  \addtolength{\taboverhang}{-1pt}

\newcommand{\tablesubheading}[2]%
{\multicolumn{#1}{|>{\columncolor{tablesubheading}}l|}{\strut\em~~~~~#2}}

% \standardtableheading produces a standard heading for tables
% field name | type | description
\newcommand{\standardtableheading}[1]%
{\rowcolor{tableheading}#1\\}

% \recordtableheading produces a standard heading for record tables:
% field name | type | description
\newcommand{\recordtableheading}
{\standardtableheading{\sl field name & \sl type & \sl description}}

% \statetableheading produces a standard heading for state tables:
% field name | type | default | description
\newcommand{\statetableheading}
{\standardtableheading{\sl field name & \sl type & \sl default & \sl description}}


\arrayrulecolor{tableheading}

% \recordtableentry{name}{type}{desc} produces one line
% of a record layout table
\newcommand{\recordtableentry}[3]
{\color{tabletext}\qq{#1} & \qq{#2} & #3\\}

% \statetableentry{name}{type}{default}{desc} produces one line
% of a state record layout table
\newcommand{\statetableentry}[4]
{\color{tabletext}\qq{#1} & \qq{#2} & \qq{#3} & #4\\}


\markboth{\dotfill MeqTree Browser Interface}{\small\tt
LOFAR/Timba/doc/MEQ/PythonInterface.tex\em~~$\circ$~$ $Revision$ $~$\circ$~$ $Date$ $\dotfill}

\title{{\sf The MeqTree Browser Inteface\\\small or how does a Python eat a
Timba?}}

\author{{\sf O.M. Smirnov}}

\date{\vspace{2cm}\small CVS path: \tt LOFAR/Timba/doc/PythonInterface.tex\\\rm
$ $Revision$ $\\$ $Date$ $}

\begin{document}

\sloppy

\maketitle
% \tableofcontents


\chapter{Writing Browser Plug-ins}

  A plug-in is a class that extends the browser functionality. Several kinds of
  plug-ins are available:
  
  \begin{description}
  
  \item[Viewer plug-ins] (e.g. Record Browser, Result Plotter) are associated
  with data types, and are activated when a user asks to display a data object
  (e.g. from the ``Display with'' context menu.)

  \item[Node actions] (e.g. MeqParm Fiddler, Reexecutor) are associated with
  nodes (all nodes, or perhaps specific classes), and are available via the node
  context menu.

  \item[Other plugins] (to be implemented) are not associated with a specific
  data object or node, and can be activated via the main menu.

  \end{description}
  
  A plug-in is generally just a Python class/module which is imported into the
  browser. All plug-ins are free to create and display any kind of PyQt widget,
  and can interact with the kernel via the {\tt meqds} (Meq Data Services)
  module. Getting a plug-in to fully integrate into the browser (i.e., 
  play nice with the gridded workspace, etc.) takes a few extra incantations,
  however.
  
\section{Viewer plugins}

  Viewers are associated with data object. The lifecycle of a viewer plug-in is
  typically as follows:
  
  \begin{enumerate} \item The user asks to display a data object with a specific
  or default viewer.

  \item The browser creates a \tt{Timba.Grid.DataItem} object. A \tt{DataItem} 
  represents the data object being viewed. The \tt{Timba.Grid.addDataItem()} 
  function is called to register the data item with the grid.
  
  \item Within {\tt addDataItem()}, a viewer object is constructed, and assigned
  to {\tt dataitem.viewer_obj}. The viewer can then allocate one or more cells
  in the grid, etc. (But note that it doesn't have to -- it's perfectly possible
  to have viewers outside the grid.)

  \item The browser maintains a list of currently active dataitems. Whenever one
  is updated (e.g., node publishes state, etc.), the corresponding viewer
  object's {\tt set\_data()} method is called to update the viewer.
  
  \item When the user closes a viewer (e.g. with the cell close button),
  \tt{Timba.Grid.removeDataItem()} is called. This deletes the dataitem and the
  viewer object.

  \end{enumerate}
  
\subsection{Anatomy of a typical viewer}

  Let's examine a viewer plug-in, the {\tt ArrayBrowser}. This class can be
  found in {\tt Plugins/array\_browser.py}. This is a very simple viewer, in that
  it allocates itself a single grid cell, and does not directly interact with
  the kernel at all, relying on the grid to keep it updated. Viewers like this
  can be conveniently derived from the {\tt GriddedPlugin} class (see {\tt
  GUI/browsers.py}). 

\begin{verbatim}  
from Timba.dmi import *           # basic DMI support
from Timba import utils           # various utility classes
from Timba import dmi_repr        # to-string conversion functions
from Timba.GUI.pixmaps import pixmaps  # icons
from Timba import Grid            # Grid services
from Timba.GUI.browsers import \
  GriddedPlugin,PrecisionPopupMenu  # 
from qt import *                  # Qt widgets
from qttable import *             # QTable and company
\end{verbatim}

  This imports various modules. 
  
\paragraph{Debug printing.} Further down, we see:
  
\begin{verbatim}  
_dmirepr = dmi_repr.dmi_repr();

_dbg = utils.verbosity(0,name='array_browser');
_dprint = _dbg.dprint;
_dprintf = _dbg.dprintf;
\end{verbatim}
  
  The {\tt dmi\_repr} object is not important, it's simly a helper object that
  the Array Browser uses to convert numbers to strings. The {\tt \_dbg} object
  is more intersting: this is a {\em verbosity context}. Each verbosity context
  has a name and a debug level (in this case initialized to 0 by default).
  All over the browser code, you will see statements such as:

\begin{verbatim} 
    _dprint(3,"rank is",self._rank);
\end{verbatim}

  This statement will conditionally print a diagnostics line on stdout, but only
  if the debug level of the corresponding verbosity context is greater than or
  equal to the first argument to {\tt \_dprint}. The debug level can be set from
  the command line via the {\tt "-d"} option, e.g.: {\tt "-darray_browser=3"}.
  This is a handy facility for adding debug printing with various levels of
  detail to your code, and you should try to use it as much as possible (in
  fact, loose {\tt print} statements are very much frowned upon...) The {\tt
  \_dprintf} function is similar, but it takes a format string as the second
  argument, e.g.:

\begin{verbatim}  
    _dprintf(3,"rank is %d",self._rank);
\end{verbatim}

  What makes this facility even more convenient is that {\tt \_dprint}
  automatically prefixes each line by the context name, filename, line number
  and function name. The statement above actually prints something like:

\begin{verbatim}  
    array_browser(array_browser.py:43:set_array): rank is 2
\end{verbatim}

\paragraph{The viewer itself.} Next, we see:

\begin{verbatim}  
class ArrayTable(QTable):
# blah blah blah
\end{verbatim}

  This is a specialization of the QTable widget for efficiently displaying
  numarrays. This is the main widget of the Array Browser, but its internals are
  not terribly relevant to this discussion, so we won't describe it (although
  you're free to use it in your plugins too). The viewer class itself is defined
  further down:
  
\begin{verbatim}  
class ArrayBrowser(GriddedPlugin):
  _icon = pixmaps.matrix;
  viewer_name = "Array Browser";
  def is_viewable (data):
    try: return 1 <= data.rank <=2;
    except: return False;
  is_viewable = staticmethod(is_viewable);
\end{verbatim}

  This derives the viewer from \qq{GriddedPlugin}, and defines an icon for the
  class, and an official viewer name (which will appear in menus and such).  The
  \qq{is\_viewable()} method is used to limit the viewer to specific kinds of
  data objects. Note that viewers are associated with specific data types (more
  on this below), so the browser already knows that \qq{ArrayBrowser} only
  applies to \qq{numarray}s. However, not all numarrays are viewable with this
  class: only one- and two-dimensional ones. The \qq{is\_viewable()} method
  checks for this\footnote{the exception handler is there as a fallback, in case
  data doesn't have a rank attribute at all, which is not supposed to happen but
  that's no reason to fall over...}; the browser will call this method and omit
  the viewer from the "Display with" menus of data objects that are not
  supported. If a viewer class can display any object of its type, then it
  doesn't need to define \qq{is\_viewable()} at all.
  
  All viewers must define a certain form of constructor. This is documented in
  \qq{Grid/Services.py}.
  
\begin{verbatim}  
  def __init__(self,gw,dataitem,cellspec={},**opts):
    GriddedPlugin.__init__(self,gw,dataitem,cellspec=cellspec);
    self._arr = None;
    self._tbl = ArrayTable(self.wparent());
\end{verbatim}

  in this case we don't really need to worry about most of the arguments, since
  they're passed up to the parent class as is. Note the \qq{self.wparent()} call
  though -- the \qq{GriddedPlugin} constructor allocates a grid cell for us
  (in fact, this is done even further up, in the \qq{CellBlock} class from
  which \qq{GriddedPlugin} inherits). 

    self.set_widgets(self.wtop(),dataitem.caption,icon=self.icon());
    if dataitem and dataitem.data is not None:
      self.set_data(dataitem);
    context_menu = self.cell_menu();
    QObject.connect(self._tbl,PYSIGNAL("fontChanged()"),self._tbl.reset_colsizes);
    if context_menu is not None:
      context_menu.insertSeparator();
      menu = PrecisionPopupMenu(context_menu,prec=self._tbl.get_prec());
      context_menu.insertItem(pixmaps.precplus.iconset(),'Number format',menu);
      QWidget.connect(menu,PYSIGNAL("setPrecision()"),\
                      self._tbl.set_precision);
      
\end{verbatim}
\begin{verbatim}  
  def wtop (self):
    return self._tbl;
\end{verbatim}


  

  
    
  
\section{Viewr  


\end{document}
