\documentclass[10pt]{article}

\oddsidemargin=-5mm
\oddsidemargin=-5mm
\topmargin=-5mm
\textwidth=170mm
\textheight=240mm

\title{MeqTree C++ Interfaces \& Data Structures}
\author{O. Smirnov}

\begin{document}
\sloppy

\newcommand{\url}[1]{{\tt #1}}

\maketitle

% \qq{text} used to quote code (in fixed font)
\newcommand{\qq}[1]{{\tt #1}}

\newcommand{\Request}{{\tt Request}}
\newcommand{\RequestId}{{\tt RequestId}}
\newcommand{\Result}{{\tt Result}}
\newcommand{\VellSet}{{\tt VellSet}}
\newcommand{\Cells}{{\tt Cells}}
\newcommand{\Vells}{{\tt Vells}}
\newcommand{\Domain}{{\tt Domain}}
\newcommand{\Node}{{\tt Node}}
\newcommand{\Parm}{{\tt Parm}}
\newcommand{\Polc}{{\tt Polc}}
\newcommand{\RES}[1]{{\tt RES\_#1}}


\section{Introduction}

The purpose of this document is to provide a C++ analogue of the MeqTree
handbook. I will try to keep it updated with design decisions we make in our
meetings. The document is available via CVS at
\qq{LOFAR/doc/MEQ/CppDataStructures.tex}.

\section{Data structures} 

\subsection{Basic principles}

  All data structures used in the Meq {\em interface} must be mappable to data
  structures of the scripting language of choice. Currently, this is Glish, in
  the future this will be Python; we should in any case restrict ourself to
  common primitives which we can expect to be available in any mature scripting
  language. Thus, we'll define ``legal objects'' as:

  \begin{itemize}
  
  \item scalars (bool, integer, float, double, float or double complex);
  
  \item strings;
  
  \item multidimensional arrays of scalars;
  
  \item lists of legal objects (in Glish this is represented either  via a
  vector of scalars or strings, or via a record with fields indexed by number);

  \item records of legal objects (a.k.a. dictionaries/maps/hashes with a string
  key and a legal object value).

  \end{itemize}

  On the C++ side, data objects are based on the DMI \qq{DataRecord},
  \qq{DataField} and \qq{DataArray} classes. Most data classes are in fact
  derived from \qq{DataRecord}, and are essentially a record with some
  predefined structure, and some extra ``shortcut'' methods. 

\subsection{Naming conventions}

  \begin{itemize}

  \item In C++, standard data structures \& nodes reside in \qq{namespace Meq}.
    In Glish, corresponding object constructors are placed into the {\tt meq}
    ``namespace'' (actually just a record), and names are all-lowercase. The
    DMI dynamic type system uses a {\tt Meq} prefix. Thus, the
    \qq{Meq::\Request} class in C++ is registered as a ``\qq{MeqRequest}'' in
    the DMI type system, and  has a  \qq{meq.request()} counterpart in Glish.

  \item The C++ side uses DMI HIIDs (hierarchical IDs) extensively. When seen
    as a string, HIIDs take the form ``\qq{Request.ID}''. When used as record
    field names on the Glish side, the same HIID takes the form
    \qq{request\_id}. We will use both forms interchangably here. Note that
    HIIDs are not case-sensitive.
    
  \item Even though the languages we use are case-sensitive, we should not rely
    on character case to distinguish identifiers. Note that C++ and Glish have
    different capitalization conventions -- e.g., \qq{Meq::Request} as opposed
    to \qq{meq.request}. In general, we should avoid situations where different
    entities have names that differ only in character case.

  \end{itemize}

\subsection{Glish \qq{meq} objects: meqtypes.g}

  With a couple of exceptions, Meq objects are represented by Glish records of
  a [mostly] predefined layout. The Glish/C++ conversion layer uses a few
  ``magic'' attributes to distinguish these objects from ordinary records, so
  it is able to map them to specialized Meq C++ classes rather than generic
  \qq{DataRecord}s. 

  Specifically, the \qq{::dmi\_actual\_type} attribute is set to a string which
  gives the DMI object type. Thus, it is possible to construct a record in
  Glish, tag it with \qq{::dmi\_actual\_type}, pass it thorugh the Glish/C++
  layer, and have it auto-magically converted to an C++ object of the
  appropriate class. The only requirement is that the record contain the
  correct set of fields, which are mapped to class attributes (data members) in
  C++. The file \qq{meq/meqtypes.g} defines  some Glish ``constructor''
  functions which create properly formed records:

  \begin{description}
  
  \item[\qq{meq.domain()}] creates a Domain object (actually a vector
    of four doubles: $f_1,f_2,t_1,t_2$.)

  \item[\qq{meq.cells()}] creates a Cells object (record).

  \item[\qq{meq.requestid()}] creates a request ID from individual components
  (domain ID, config ID, iteration ID). A request ID is a \qq{HIID}.
  
  \item[\qq{meq.request()}] creates a Request object (record).
  
  \item[\qq{meq.polc()}] creates a Polc object (record).
  
  \end{description}
  
  Some Meq classes have Glish counterparts but no constructor functions, since
  these objects are passed from C++ to Glish but not vice versa (it is always
  possible to implement such constructors in the future, if necessary). These
  include \Vells, \VellSet{}s, and \Result{}s.

  In general, all data objects on the C++ side have counterparts on the Glish
  side. Within this document, we will describe data objects in terms of Glish
  records and record fields, with the understanding that there is a trivial
  mapping from that to C++ classes and data members.

  Several other functions in \qq{meqtypes.g} help to construct defrecs for
  nodes (see below). \qq{meq.node()} puts together a basic node defrec -- class
  name, node name, optional children specification, optional node group
  list. \qq{meq.parm()} is a specialization for constructing \Parm\ defrecs.
    
\section{MeqNode}

  The abstract base class \qq{Meq::Node} implements the basic node behaviour:

  \begin{itemize}

  \item A node may have a number of child nodes. Generally, a node has no
    knowledge of the types of its children. Subclasses may assign formal child
    labels (akin to argument names) to specify semantics, or may leave their
    children unlabeled. Child labels are assigned via the constructor of the
    subclass.

  \item To nail down directional terms: root nodes and parents are at the {\em
    bottom}, children and leaves are at the top. 

  \item Each node is assigned a unique index (integer$>$0) and an optional name. A
    MeqForest object acts as a repository of nodes, and maintains a map between
    names, indices and node objects.

  \item \Node\ has an \qq{execute()} method, taking a \Request\ parameter,
    and returning a \Result. Normally, a node is expected to call execute()
    with the same \Request\ on its children, and form its result based on the
    results of its children. Thus, requests propagate up the tree, and results
    percolate down the tree.

\begin{verbatim}
    virtual int execute (Result::Ref &result,const Request &request);
\end{verbatim}

    \Result{}s are returned by attaching them to the CountedRef. The return code
    of \qq{execute()} is significant; see details below.

  \end{itemize}

\subsection{Initialization and state}
  
  A node's full {\em state} should be mapped to a {\em state record}. State may
  be requested via \qq{getState()} and changed via \qq{setState()}. Note that
  the argument to \qq{setState()} does not have to be a complete new state
  record; instead, it should contains only those fields that actually need to
  be changed. 

  When a node is constructed, it is passed an {\em init-record} (via the
  \qq{init()} method), which is essentially a complete initial state record.
  This is the way that all run-time arguments to a node are specified! Later 
  in a node's lifetime, the \qq{setState()} method may the be called to
  reconfigure  it. Note that a node class is not obliged to be reconfigurable
  in every single aspect, but it's good design to make it so as much as
  possible. If some of the node state may only be set once via \qq{init()} and
  not changed later on via \qq{setState()} -- call this {\em static\/} state --
  it should be clearly documented as such. The assumed default is {\em
  dynamic\/} state, i.e., state that is freely reconfigurable via
  \qq{setState()}.

  The following methods are responsible for initializing and changing state:

\begin{verbatim}
  // public: Initializes node with init record
  //         Note that Ref::Xfer implies that ref to record will be taken over
virtual void init (DataRecord::Ref::Xfer &initrec);

  // public: Changes dynamic node state (note: non-virtual)
  //         Node can attach to/take over record contents as needed.
void setState (DataRecord &rec);

  // protected: Checks init record for missing fields, fills in defaults where needed
  //            (called from Node::init())
virtual void checkInitState (DataRecord &rec);

  // protected: Implementation for setting or changing internal dynamic state 
  //            (called from Node::setState())
  //            Node can attach to record contents as needed. If initializing,
  //            then record is the state record and should not be changed. If
  //            not initializing, node can take over contents as well.
virtual void setStateImpl (DataRecord &rec,bool initializing);
\end{verbatim}

\subsubsection{init()}

  The base \qq{Node::init()} does the following:

  \begin{enumerate}
  
  \item Takes over the init record,  sets it as the state record, ensures a
    private \& writable copy.
    
  \item Adds the node's classname (\qq{rec.class}) to the state record if not
    already present. If present, checks that the name actually matches the node
    class.

  \item Calls the virtual \qq{checkInitState()} method with the state record,
    to ensure that it's complete, and that any missing defaults are filled in.

  \item Calls \qq{setStateImpl(staterec,true)} to set up internal state from
    the state record (the \qq{true} argument indicates that the node is being
    initialized with the state record.)

  \item Looks at \qq{rec.children} to set up child nodes. Children may be
    specified via a either a list or a record, containing any mix of the
    following:

    \begin{itemize}
      \item integer node indices referring to existing nodes.

      \item string node names, referring to existing or yet-to-be-created
        nodes. In the latter case, \qq{Node::resolveChildren()} must be called
        later on; this will recursively resolve all children not found at
        \qq{init()} time.

      \item init-records, to recursively create child nodes on-the-fly (each of
        these records must have a field named \qq{class}, containing the child
        node classname).

    \end{itemize}

    If a list is specified, then the children are simply attached in the given
    order. If a record is specified, then the field names must match the child
    labels set up by the subclass (if no labels are set up, the record is
    processed just like a list, with no specific order ensured). The record
    form allows for a more formal child specification in the case where
    children are semantically different.
    
  \item Any errors will result in an exception being thrown at the caller. A
    node object that fails \qq{init()} is under no obligation to be usable; the
    only method that's not allowed to fail is the destructor.

  \end{enumerate}

  Derived classes need to reimplement \qq{init()} only if they have additional
  static state of their own. A derived \qq{init()} should do the following:

  \begin{enumerate}
  
  \item Call the parent class's \qq{init()} with the initrec. This should
    ultimately call \qq{Node::init()}, thus setting up the state record  and
    calling \qq{setStateImpl()} to set up dynamic state.

  \item Set up static state, as defined at the child class level, in accordance
    with the state record.

  \item Throw exceptions on any error. 

  \end{enumerate}
  
  Note the virtual \qq{checkInitState()} method is called from
  \qq{Node::init()}. This is meant to check the init record for required
  fields, and fill in any missing defaults. The following two macros/inlines
  (defined in \qq{Node}) are handy for this:

  \begin{itemize}
  
  \item \qq{requiresInitField(record,field)} will throw an exception if the
    field is missing.

  \item \qq{defaultInitField(record,field,deflt)} will insert a default value
    for a field if it is missing.

  \end{itemize}
  
  The base \qq{Node::checkInitState()} only fills in a default for \qq{.name}
  (empty). A derived \qq{checkInitState()} should call the parent version, then
  check for additional defaults and required fields as defined by the child
  class.

\subsubsection{setState() and setStateImpl()}

  The non-virtual \qq{setState()} method defined in \qq{Node} provides the
  public interface for setting state. Basically, it defers parsing the record
  to \qq{setStateImpl()}, while providing a transaction mechanism of sorts:

\begin{enumerate}

  \item Calls \qq{setStateImpl(rec,false)} to process the record. The
    \qq{false} value indicates that state is being modified rather than
    [re]initialized. (Note that if the supplied record happens to be the node
    state record itself, \qq{true} will be passed in instead.)

  \item Catches \qq{Node::FailWithoutCleanup} exceptions and rethrows them at
    the caller with no additional action.

  \item Catches all other execeptions, and does a cleanup before rethrowing
    them. The cleanup consists of calling \qq{setStateImpl(staterec,true)}, so
    as to reset internal state from the current state record. This is meant to
    roll back from situations where an error midway through \qq{setStateImpl()}
    could cause internal object state to decohere from the state record.

  \item On success, merges the supplied record into the current state record.

\end{enumerate}

  This design ensures that if a \qq{setState()} call fails (i.e., with an
  exception), both the state record and the internal state of the object are
  rolled back to their values prior to the call. (Assuming they were mutually
  consistent to begin with.) In other words, the node object is guaranteed to
  remain usable.

  The virtual \qq{setNodeState()} method is responsible for changing dynamic
  state. It should do the following:

\begin{enumerate}

  \item If the \qq{initializing} flag is not set, check the record for
    ``forbidden'' fields, i.e., attempts to modify static state. Throw a
    \qq{FailWithoutCleanup} if any are present. The
    \qq{protectStateField(record,field)}  macro/inline is a convenient way to
    do this.

  \item Call the parent \qq{setStateImpl()}, which will check for forbidden fields and 
    change dynamic state defined at the parent class level.
    
  \item Parse the record and modify state defined at the child class level.
    
  \item Throw exceptions on error. A \qq{Node::FailWithoutCleanup} should be
    thrown if and only if no state was modified. All other exceptions will
    invoke the ``rollback'' mechanism above. You can rely on \qq{DataRecord}
    (and other DMI classes) to throw an exception when datatypes mismatch or
    something else goes wrong; throw a \qq{Node::FailWithCleanup} if you want
    to indicate some other kind of failure.

\end{enumerate}

  For reference, the base \qq{Node::setStateImpl()} does the following:  

\begin{enumerate}

  \item Unless initializing, fails if static state: \qq{.class}, \qq{.children}
    or \qq{.nodeindex} is specified.

  \item Gets node name (\qq{.name}) from the state record.

  \item Gets the node group list from the state record  (\qq{.node\_groups},
    see below).

  \item Sets up caching policy according to what is specified in the init
    record (\qq{.cache\_policy}, details TBD).

\end{enumerate}

\subsubsection{Unrecognized state fields}

  Note that these implementations of \qq{init()} and \qq{setState()} will
  ignore any unrecognized fields in the init and state record, all the while 
  diligently maintaining and changing them as requested. This allows for a few
  interesting possibilities:

  \begin{itemize}
  
  \item Nodes may be assigned arbitary additional attributes and data,
    meaningless to the node itself, but perhaps useful to the control layer.

  \item Things like the result cache (see below) -- which is stored directly in
    the state record -- may be accessed and changed externally (i.e. from the
    scripting level). This may be useful for debugging and testing.

  \end{itemize}

\subsection{Node serialization \& persistency}

  In the near future we will need persistent nodes (i.e. being able to
  save/load the nodes to a file or database), and further down the road, the
  possibly the capability to move a node across a network. This implies being
  able to serialize a node.
  
  Serialization is implemented through DMI mechanisms. A \qq{DataRecord} is
  inherently serializable. To serialize a node, the control code will simply
  serialize its state record. To unserialize a node, it will recover the
  record, create a node object (as specified by the \qq{class} field), and call
  \qq{init()} on it. 

  Thus, subclasses of \Node\ should take care to maintain their state record
  appropriately. Basically, each node class should ensure that it is completely
  re-creatable (via \qq{init()}) from a snapshot of its state record at any
  point in time.

\section{MeqRequest}

  A \Request\ is an {\em atomic job description,} because generally speaking,
  it contains everything a node needs to know in order to produce a \Result. A
  \Request\ contains one or more of the following components:

  \begin{description}
  \item[Node state,] specifying change of state for specific nodes up the tree.
    E.g., make some parameters solvable or non-solvable, or change wsum
    settings. 

  \item[Cells,] specifying a domain \& grid over which a node is to be
    evaluated.

  \item[optional flags] \& other attributes (if needed for \Cells\ or change of
    state. E.g. the calc\_deriv flag).

  \item[optional rider] record, containing {\em commands} and state changes.
  
  \end{description}
      
  A \Request\ always has a unique \RequestId\ (see below). Whenever a different
  request is generated, it must be assigned a different id (see below).

  \Request\ is derived from \qq{DataRecord}. It has the following fields (of
  which only request\_id is obligatory).

  \noindent\begin{tabular}{lp{.8\textwidth}}
  \qq{.request\_id} & the request ID (a HIID)\\
  \qq{.cells}       & [opt] a \Cells\ object (see below)\\
  \qq{.calc\_deriv} & [opt] boolean: compute derivatives over cells? 
                      Default is false.\\
  \qq{.next\_request} & [opt] a hint of what the next request is going to be.
                      This matters for caching decisions, and for ``rippling''
                      trees.\\
  \qq{.rider}       & [opt] rider record, used for additional commands, 
                      etc. (see below)
  \end{tabular}
  
\subsection{Request IDs}

  In our discussions, it's become apparent that to make optimal use of the node
  cache during solving, what we really need is a {\em hierarchical} request ID.
  For example, the request ID could be represented by a HIID composed of three
  indices:

  {\tt\em  
  $<$cells\_index$>$.$<$config\_index$>$.$<$value\_index$>$
  }
  
  The components of the request ID describe how this \Request\ is different
  from the previous one. The domain index changes whenever a new \Cells\ is
  used. The config index changes whenever a tree is reconfigured (via, e.g.,
  changing the weights of a MeqWSum node, etc.) The value index changes
  whenever MeqParms are assigned new values (i.e., with every solve iteration).
  It is up to the generator of the \Request\ to make sure that the indices
  follow these rules. The reasoning behind this will be expounded below, in the
  discussion on caching behaviour.

  Note that this interpretation of the request ID is highly
  application-specific. As will become apparent below, it is worthwhile to
  think of the ID in more general terms, that is, as a sequence of $N$ indices,
  each one referring to a particular property of the request. 

\subsection{Next\_request and request sequences}

  The concept of a ``request sequence'' has been bandied about a lot (see the
  ``Making Waves'' document, plus various discussions). The reason for this is
  that a node needs some capability of ``looking ahead'' to future requests, in
  order to efficiently utilize cache, and to enable ``rippling'' trees that
  parallelize well.

  In fact, it is sufficient to be able to answer the more limited question:
  given request $X$, what is the next request likely to be? A node doesn't 
  need to know the full request sequence -- only the next step of it. The
  mechanism can be hidden behind a single method called, e.g.,
  \qq{getNextRequestHint()}, implemented at the \Node\ level. This neatly
  factors the issue into two independent ones:

  \begin{itemize}
  
  \item {\em Where does the base \qq{\em Node} class get knowledge of the
    next request?} Obviously, the originator of the \Request\ can have some
    idea.  For example, a Solver node would probably know if more iterations
    over the same domain/source are required, or if we're going to the next
    domain. A Sink node may know what the next domain is going to be. Etc. It
    seems reasonable to place this information into the \Request\ record
    itself, hence the \qq{next\_request} field.

  \item {\em Given knowledge of the next request, how does that help us in
    caching and parallelization?} Parallelization was the subject of the
    ``Making Waves'' document, while caching issues are discussed below.

  \end{itemize}
  
  The next-request hint is just that, a hint, with no commitment implied. It is
  not necessary at all for correct operation of the system -- even the hint
  itself can even be wrong. The maximum penalty to pay for an incorrect hint is
  recalculation of the tree rooted at the node in question. Efficient
  operation, however, requires that the hint be correct most of the time.

  {\em Hints and sequences are probably outside of PSS-4 scope; the point of
  the present discussion is to run through mental models of the mechanism and
  convince ourselves that we're making the right decisions.}

\section{Results}

\subsection{MeqResult}

  A \Result\ contains the result of a \Request's execution. It is derived from a
  \qq{DataRecord}, and may contain the following fields:

  \noindent\begin{tabular}{lp{.8\textwidth}}
  \qq{.cells} & [optional] the originating request's \Cells, if any.\\
  \qq{.values} & [optional] a list of \VellSet\ objects for these \Cells.\\
  etc. & any optional flags \& attributes (e.g. solver may use these to
                return the solution).
  \end{tabular}
    
\subsection{MeqVellSet}

  A \VellSet\ contains a value plus optional perturbed values for a \Cells. It
  is derived from a \qq{DataRecord}, and contains the following fields:

  \noindent\begin{tabular}{lp{.8\textwidth}}
  \qq{.value}  &  the value: a \Vells\ of the shape determined by \Cells, or 
                  scalar if constant in time/freq).\\
  \qq{.spids}  &  [opt] a list of spids\\
  \qq{.perturbations}  & [opt] a list of perturbations (same length as spids)\\
  \qq{.perturbed\_value} & [opt] perturbed values: a list (same length as spids); 
                    of array or scalar \Vells\\
  \qq{.fail}  & a list of one or more fail-records (see below)
                describing failures that have occurred\\
  etc. &        any optional flags \& attributes.
  \end{tabular}
                    
  The \qq{value} and \qq{fail} fields are mutually exclusive. A \VellSet\ with
  neither is an empty (uninitialized). A \VellSet\ with a \qq{fail} in it is
  called a fail-\VellSet\ (or simply a fail). The \qq{spids},
  \qq{perturbations} and \qq{perturbed\_value} fields are optional, but either
  none or all three must be present at the same time.
  
  Note also that this layout could be extended in the future to accomodate
  double derivatives and analytic derivatives.

\subsubsection{MeqVells}

  On the scripting side, a \Vells\ object is simply a scalar or an array. On
  the C++ side, it is also essentially an array, with some run-time type and
  size information. The current implementation supports double and
  double-precision complex types, in either scalar or 2D-array form, but this
  could be extended in the future.

  \Vells\ are covered in some more detail in the description of \qq{Function}
  nodes, below.

\subsection{Multiple values}

  The point of defining a \Result\ as a set of \VellSet{}s is to allow multiple
  return values for a single \Cells\ (a.k.a. "multiple planes"). E.g., a
  \qq{Spigot} may return four values at a time (for the four correlations).
  Function nodes expect all child \Result{}s to have the same shape (i.e., same
  number of values), and will apply the function to each "plane" independently.

  The Selector and Composer nodes can be used to decompose and assemble
  \Result{}s.
  
\subsection{Fail propagation}

  Note that a set may contain a mix of failed and normal \VellSet{}s. Failed
  values should propagate down the tree in an orderly fashion. Normally, a
  fail-value from one of its children produces a fail at the same position in
  the output \Result\ (the contents of the fail -- origin \& description -- are
  preserved.)

  For example, if a \qq{Spigot} node is configured to return four correlations,
  and the data source only has XX and YY, then XY/YXs will be represented by
  fails. These fails would propagate all the way down the XY/YX trees, to a
  \qq{Sink} node, which can handle them benignly (by not writing XY/YX data,
  for example).

\section{The details of Node::execute()}

  The virtual \qq{Node::execute()} method is responsible for processing a
  \Request. The default version decomposes the request and calls a number of
  virtual methods to handle it. All but the most exotic nodes should only need
  to implement the handlers -- but the possibility to reimplement
  \qq{execute()} itself is always there. \qq{Node::execute()} does the
  following:

  \begin{enumerate}
  
  \item Compares the request id to the id of the previous request, if any. If
    there's a cached \Result, and the request id matches the cached request id,
    then does nothing, immediately returning the cached \Result.  If IDs do not
    match, clears the cache. (Note that the match is somewhat less strict than
    literal equality. See caching issues, below, for a discussion of ID
    matching.)

  \item For new requests only: if a rider record is present, parses it and 
    calls the \qq{processCommands()} method if a command set for the node
    is found.

  \item If node has children, calls the virtual \qq{pollChildren()} method to
    collect results from its children. The default implementation of this
    method calls \qq{execute()} with the same \Request\ on all the child nodes,
    collecting their \Result{}s into a vector of \qq{Result::Ref}s, and
    computing the {\em cumulative result code} as a bitwise-OR of the child
    return values. If any child returns a \RES{FAIL}, it creates an output
    \Result\ containing a merge of all the fail-results.

    If the cumulative result code contains \RES{WAIT} or \RES{FAIL},
    \qq{execute()} returns immediately with that result code. 
    
    Note that the default implementation of \qq{pollChildren()} is appropriate
    for most node classes, with the exception of specialized ``control'' nodes
    such as \qq{Sink} and \qq{Solver}.

  \item If a \Cells\ object is present, calls the virtual \qq{getResult()}
    method, passing in the vector of child \Result{}s returned by
    \qq{pollChildren()}. Bitwise-ORs the return value with the cumulative
    result code.

  \item If an exception is thrown at any stage of the process, \qq{execute()}
    will catch it, create an output \Result\ with a fail-result describing the
    exception, and return \RES{FAIL}. Thus, an exception is the normal way for
    \qq{setState()} or \qq{processRider()} to indicate failure.
    
  \item Upon exit, optionally stores the output \Result\ and result code in the
    cache (see caching issues, below).

  \end{enumerate}

  The return value of \qq{execute()} is simply the accumulated result code.
  
\subsection{Result codes}

  The return value of \qq{execute()} is meant to be a result code describing
  certain properties  of the returned \Result. It is a bitmask composed of a
  number of flags listed below. Each flag describes a certain property of the
  \Result. The property semantics are defined in such a way that, in most
  practical cases, a flagged property in any child result is inherited by the
  parent's result. This allows \qq{execute()} to accumulate the correct result
  code via a simple bitwise-OR. In other words, only a few classes (Parm, Time,
  Freq, Spigot, and perhaps some special nodes) have any specific logic for
  determining result codes. For all other classes, the correct result code is
  simply the bitwise-OR of their children's codes. The following bits are
  defined:
  
  \begin{description}
  
  \item[\RES{UPDATED}:] result has changed from that of previous \Request. This
     is meant as a caching hint; if you're not sure, raise the flag.

  \item[\RES{DEP(0)}, \RES{DEP(1)}, etc.:] result will change if the $i$-th
    index of the request ID changes. For calibration \& solving applications,
    request ID has a specific form (see above), and the following aliases may
    be defined:

    \begin{description}
    
    \item[\RES{DEP\_DOMAIN = RES\_DEP(1)}:] result will change if domain/cells
      change. True for all Spigots and all Parms having a time/freq dependence,
      false for constant Parms.

    \item[\RES{DEP\_CONFIG = RES\_DEP(2)}:] result will change if configuration
      changes. True for nodes like WSum.

    \item[\RES{DEP\_VALUE = RES\_DEP(3)}:] result will change if parameter values
      change. True for solvable Parms, false for all others.

    \end{description}
    
    Note that it is only the leaf nodes (which tend to be application-specific)
    and the \Request\ originator (ditto) that care about specific meanings of
    the request ID sub-indices and specific \RES{DEP} flags. All other nodes
    can ``think'' of it in abstract terms, i.e., with $N$ sub-indices of the
    request ID mapping to flags (see discussion on caching, below).

  \item[\RES{DEP\_VOLATILE} = RES\_DEP(0):] result may change in response to
    external events, even without new requests. (This is not implemented for
    now, and only meant as a placeholder for future developments, such as
    growing domains, partial integration, etc.)

  \item[\RES{FAIL}:] result is a ``complete'' fail. Note that this is not the
    same thing as a result containing some mix of valid and failed VellSets;
    rather, this indicates a failure for the whole result overall. \RES{FAIL}s
    are usually generated when a node runs into some unrecoverable error.
    
    When this flag is returned, a \Result\ object is expected; it should
    contain one or more fails describing the error. Note that dependency flags
    can be meaningfully combined with \RES{FAIL}.

  \item[\RES{WAIT}:] no result available, wait for notification or try later.
    If this flag is raised, then no \Result\ should be returned. Note that
    other flags can and should be meaningfully combined with \RES{WAIT}, since
    it usually possible to indicate the behaviour of a node in advance.
    
  \end{description}
  
  The return value of a node's \qq{getResult()} method should describe any {\bf
  additional} properties introduced by that node. Most function nodes will
  return zero, indicating that a node does not introduce any additional
  dependencies into the result. As a counter-example, the \qq{UVW} node (which
  gets the RA/Dec and ITRF station positions from its children) will return
  \qq{RES\_DEP\_DOMAIN}, since $UVW$ coordinates also depend on time.

\subsection{Caching issues}

  To avoid unnecessary recalculations, a node's result can be retained in a
  cache. Obviously, this trades off performance against memory footprint. Three
  broad caching policies have been identified so far:

  \begin{description}
  
  \item[Never:] no caching at all, values are always recalculated anew. 
   
  \item[Always:] result is always cached, until a different request comes in.
    (This is the policy implemented currently). While expensive in terms of
    memory, this can be very useful for debugging, since it allows one to pause
    the system and examine the most recent result of every node.

  \item[Smart:] result is cached according to memory availability, expected
    request sequence, etc.

  \end{description}
  
  Caching policy should be settable on a per-node basis, via the init record
  (and changed via \qq{setState()}). The cache itself is simply part of the
  node's state record (and thus can even be changed manually if needed.)
  
  The ``Smart'' policy is the really interesting one. The combination of result
  codes, hierarchical request IDs and next-request hints allows us to get
  pretty smart. 
  
  Let's call the individual components of a request ID {\em dependencies} --
  because the \RES{DEP} flags in the result code describe what the result
  depends on. Note that a parent result inherits dependencies from all its
  child results, thus the parent dependencies are always a superset of any
  child dependencies (i.e. equal to or larger.) The following simple rules seem
  reasonable:

  \begin{enumerate}
  
  \item If a dependency will change in the next request, then node can clear
    cache (since the next request will invalidate it anyway).

  \item If the parent dependencies are exactly equal to those of a child, then
    the parent does not need the child's cache (because any request requiring a
    recalculation of the parent result will also require a recalculation of the
    child result.)

  \item If the parent dependencies are larger than those of a child, then the 
    child should retain cache (unless [1] holds for the child.) 

  \item A child must retain cache unless [1] is true, or {\bf all} its parents
    have told it to release cache.

  \end{enumerate}
  
  Note also that when a node retains the result in a cache, it also retains the
  dependency mask and the request ID. When deciding whether to return the
  cached result or recalculate the request (see step [1] of \qq{execute()}),
  the request IDs are matched according to the dependency mask.
  
\subsection{Commands in request riders}

  The optional \qq{rider} sub-record of a \Request\ may be used to send
  additional commands to specific nodes. The Solver node, for example, makes
  heavy use of this feature to implement iterative solutions. Command sets are
  specified via records, with the field names being the commands per se, and
  the field values being the command arguments. If the record contains multiple
  commands, they are processed in a specific order. Some example commands are:

  \begin{description}
  
  \item[\qq{state}:] to change the state of a node (available for all nodes);
  
  \item[\qq{set\_value}:] to change the values of a polc in a MeqParm (see
  below);
  
  \item[\qq{save\_polc}:] to save the polc(s) in a MeqParm (see below).
  
  \end{description}
  
  The rider record is structured so that it is possible to associate a command
  with a specific node or a set of nodes. A node will check if the request
  contains any commands for itself. Note that in a very large tree, these
  repeated checks may become expensive. It is for this reason that we introduce
  the concept of {\em node groups}. 
  
  A node may be associated with one or more groups. Groups are specified via a
  list of HIIDs passed in the \qq{.node\_groups} field of the init record (they
  can subsequently be changed via \qq{setState()}). All nodes automatically
  belong to the \qq{all} group. A node's groups are used as a first-level index
  into the \qq{rider} record. If a node belongs to groups \qq{foo} and
  \qq{bar}, then \qq{Node::execute()} will check for the {\em command
  subrecords}\/ (CSRs) \qq{rider.foo}, \qq{rider.bar}, and \qq{rider.all}, in
  that order, and process any subrecords that it finds.

  Each CSRs contains a number of command sets. These command sets may be
  associated with specific nodes. There are three ways to specify these
  associations:
  
  \paragraph{All nodes in group:} The command sets specified via the field
  \qq{command\_all} is applied to all nodes in the group. For example:

\begin{verbatim}
  - req.rider.foo.command_all
  [ save_polc=T,state=[solvable=F] ]  
\end{verbatim}

  ...will call \qq{processCommands()} on all nodes in group \qq{foo} (saving
  polcs and setting their state to non-solvable.)

  \paragraph{Via node index:} Command sets may be associated with a specific
  node index. This is done via field \qq{command\_by\_nodeindex}, which is
  essentially a map from node index to command set. For example:

\begin{verbatim}
  - req.rider.foo.command_by_nodeindex
  [ #19 = [ value=1,save_polc=T ], #41 = [ value=2,save_polc=T ] ]
\end{verbatim}

  ...will cause a \qq{processCommands()} call on nodes 19 and 41. (Note that
  since Glish only supports strings for record field names, the \qq{'\#ddd'}
  form is used to specify a ``numeric'' node index.)

  \paragraph{Via lists:} The third way is to associate command sets with nodes
  listed by name or node index. This is specified via field
  \qq{command\_by\_list}. For example:

\begin{verbatim}
  - req.rider.foo.by_list
  [ *1 = [ name="RA DEC",nodeindex=[17,32],state=[solvable=T],command=[save_polc=T] ],
    *2 = [ state=[solvable=F] ]
\end{verbatim}

  ...will call \qq{processCommands()} with the first command set on nodes RA,
  DEC, \#17 and \#32, and with the second command set on all other nodes in
  group \qq{foo}. To be more specific, the \qq{command\_by\_list} field is
  treated as a list  of records. Each record in the list is a command set, with
  an additional \qq{name} field (string or vector of strings) and/or a
  \qq{nodeindex} field (integer or vector of integers). \qq{Node::execute()}
  will iterate through the records one by one; if the node's name is found in
  \qq{name}\footnote{In the future, pattern matching will be supported as
  well.}, or the node index is found in \qq{nodeindex}, then
  \qq{processCommands()} is called with the contents of that record. Once a
  match is found, list processing stops. As a special case, if neither
  \qq{name} nor \qq{index} is specified, then the entry is a ``wildcard''
  matching any node. Wilcards are only useful at the end of the list, to catch
  nodes not already matched by previous entries.
  
\subsubsection{Command evaluation order}

  In cases where a node finds more than one command set associated with it, the
  order of processing becomes important. The rider is parsed in the following
  order:

  \begin{itemize}
  
  \item The outer loop is over node groups, in the order in which they are
  specified in the node state record. The \qq{all} group is checked last.
  
  \item Within a group's command sub-record, the processing order is from least
  specific to most specific: \qq{command\_all}, \qq{command\_by\_list},
  \qq{command\_by\_nodeindex}.
  
  \item Once a command set is passed to a node's \qq{processCommands()} method,
  the order of processing is determined by the node class implementation.
  Generally, a subclass should call its parent's \qq{processCommands()} first,
  so general commands (such as \qq{state}) should be processed before
  more class-specific commands (such as \qq{save\_polc}).
  
  \end{itemize}

\section{Description of specialized nodes}

\subsection{MeqParm}

  The \Parm\ node implements a possibly solvable Measurement Equation
  parameter. The parameter is represented by one or more \Polc\ objects.

\subsubsection{MeqPolc}

  The \Polc\ class implements a 2D polynomial in time and frequency. The Glish
  equivalent is a \qq{meq.polc} record. This record will contain the following
  fields:
  
  \noindent\begin{tabular}{lp{.8\textwidth}}
  \qq{.coeff}  &  a 2D array of polynomial coefficients.\\
  \qq{.freq\_0}  &  \\
  \qq{.freq\_scale}  &  \\
  \qq{.time\_0}  &  \\
  \qq{.time\_scale}  & the scale of the polc (see below)\\
  \qq{.domain}  & (optional) the polc \Domain.\\
  \qq{.weight}  & weight\\
  \qq{.pert}  & perturbation to use for computing derivatives\\
  \qq{.dbid\_index}  & database ID (see below)\\
  \qq{.inf\_domain} &  optional flag: infinite domain (see below)\\
  \qq{.grow\_domain} &  optional flag: growing domain (see below)\\
  \end{tabular}
  
  The value of a polc for frequency $f$ and time $t$ is computed as follows:

  \begin{equation}
  p(f,t) = \sum_{i=0}^{N-1}\sum_{j=0}^{M-1} c_{ij}(\frac{f-f_0}{s_f})^i(\frac{t-t_0}{s_t})^j
  \end{equation}
  
  Here, $c_{ij}$ is an $N\times M$ array of coefficients (\qq{coeff}), and
  $f_0,s_f,t_0,s_t$ is the scale of the polc. The scale is only necessary to
  keep numbers small so as to avoid round-off errors.

\subsubsection{MeqParm state}

  Here's the layout for a \Parm\ state record. All of these attributes are
  dynamic state, that is, they may be changed at any time via \qq{setState()}:
  
  \begin{description}

  \item[\qq{solvable}] (bool, optional) is this \Parm\ solvable or not?
    Default is non-solvable.

  \item[\qq{polcs}] (list of \Polc{}s, optional) the parm's polc list (see below).

  \item[\qq{table\_name}] (string, optional) the name of a MEP table. Default is none.
    If no table is provided, then polcs must be specified via either the
    \qq{polcs} or the \qq{default} field.

  \item[\qq{parm\_name}] (string, optional) the name to use when working with a MEP
    table. Default is to use the node name itself.

  \item[\qq{default}] (\Polc, optional) a default \Polc, to be used as a last
    resort (i.e. if no polcs found in the MEP table).

  \item[\qq{auto\_save}] (bool, optional) If true, any updates to the parm's 
    polcs are immediately saved to the MEP table. If false, updates need to be
    explicitly saved (more on this below). Default is false.

  \end{description}
  
\subsubsection{Selecting polcs}

  The \qq{Parm::getResult()} method is responsible for evaluating a parm over a
  \Cells\ (specifically, over the domain of the \Cells). To do this, the parm
  needs to find an appropriate polc or polcs for the given domain. Most of the
  tricky logic of the \Parm\ class is dedicated to choosing the right set of
  polcs.  Currently, it will select polc(s) as follows:

  \begin{enumerate}
  
  \item If no MEP table and no default polc has been set in the state record,
    then the parm will blindly re-use the current polc list (i.e. the
    \qq{polcs} field of the state record), or fail if the list is empty.  No
    further error checking is done, and it is up to the user to ensure that the
    parm has been initialized with polcs that are meaningful w.r.t. the domain
    of the request.

  \item Otherwise, the parm will first see if it can re-use the current polc 
    anyway. The current list must contain a single polc, which is tested for
    re-usability as follows:

    \begin{enumerate}
    
    \item If the \qq{inf\_domain} flag is set, the polc has an infinite domain, 
      and can be re-used.
      
    \item If the \qq{grow\_domain} flag is set, and the requested domain is a
      superset of the polc's current domain, then the domain of the polc is
      expanded to match the requested domain, and the polc can be re-used. 

    \item If the requested domain is a subset of the polc's current domain, the
      polc is re-used.

    \end{enumerate}

  \item If no re-use is possible, the current polc list is cleared. The parm
    will then query its MEP table for polcs whose domain overlaps the requested
    domain. These polcs are loaded into the polc list.

  \item If no polcs are found in the MEP table, the table is checked for a
    default polc. Note that default polcs have no domain (generally, they will
    only have a $c_{00}$ coefficient). If a default polc is found, it is placed
    in the list, and its domain is set to the requested domain.

  \item If no default polc is found in the table, or if no table is available,
    then default polc from the state record is copied into the list, and its
    domain is set to the requested domain. Note that the case of no table and
    no default polc is covered by (1), above.

  \end{enumerate}
  
  Once the selection is complete, the parm will have at its disposal a list of
  one or more polcs. 

\subsubsection{Selecting a single solvable polc}
  
  If the parm is set to solvable, the polc list is then culled to a single polc
  (NB: future versions may support solving for multiple polcs). This polc is
  selected according to the following criteria (in descending order of
  importance):

  \begin{itemize}
  
  \item an exact match of the requested domain to the polc domain;
  
  \item weight (higher is better);
  
  \item database ID (higher, i.e. more recent, is better).
  
  \end{itemize}
  
  Once a single solvable polc is selected, its domain is set equal to the
  requested domain.
  
\subsubsection{Evaluating the polc list}

  A single polc is evaluated directly over the \Cells\ of the incoming
  \Request. The resulting \VellSet\ object (polc value, plus optional
  derivatives) is then returned to the caller. 
  
  If the parm contains multiple polcs, different evaluation schemes are
  possible:

  \begin{description}

  \item[Tiled polcs.] This is the only scheme implemented by \Parm\ at time of
  writing. For every polc in the list, \Parm\ computes the overlap between
  the polc's domain and the domain of the request. The polc is then evaluated
  over that part of the \Cells\ grid which falls within the overlapping area.
  This step is repeated for every polc. Sections of the grid with no polc
  domain coverage are assigned a zero value.

  Obviously, this scheme is most suitable when the polcs neatly tile
  the request domain. Note that the result is generally discontinuous across
  tile boundaries, which can be quite useful for representing things such as
  phase jumps. On the other  hand, if the request domain is not completely
  covered, or if the polc domains overlap, the results of this scheme are not
  very well defined. 

  \item[Weighted mean.] Each polc is associated with a set of weights, defined
  on the \Cells\ grid. Different weighting schemes may be employed. Presumably,
  grid points within the polc's domain are assigned a higher weight, while
  points outside the domain get a lower weight, further decreasing as we get
  further away from the domain. The value of the parm at each grid point is
  then simply a weighted mean of all the polcs' values.

  The advantage of this scheme is that it gracefully incorporates polcs with
  overlapping domains, while allowing for extrapolation to non-covered areas of
  the request domain. The result is always smooth and continous -- which, on the
  other hand, may not always be what you want. The downside is computational
  expense, as each polc needs to be evaluated at each grid point.

  \item[Reduction to single polc.] This scheme involves fitting a single polc
  (usually of a higher order) to multiple polcs. It can already be implemented
  externally (see \qq{fitpolcs\_wlc.g} for an example), as a ``preprocessing''
  stage of sorts. The scheme works by evaluating the polcs over some set of
  points within the request domain (not necessarily the \Cells\ grid at all),
  combining the results using some sort of weighted mean, then doing a
  least-squares fit of a new polc to the resulting values.

  \end{description}

\subsubsection{Updating and saving polcs}    

  In the course of a solution, the Solver node updates the values of solvable
  polcs by sending up new values in the request rider. This is done by
  including the \qq{set\_value} command in the rider. The value of the command
  is expected to be a vector of new coefficients, in the same order in which
  spids were assigned. More specifically:
  
  \begin{itemize}
  
  \item When a \Parm\ that has been set solvable initializes a new polc for a
  domain, it associates a number of spids with its coefficients. Spids are
  assigned as $256*${\em nodeindex}$+i$, where $i$ is the number of the
  coefficient.
  
  \item These spids are included in the resulting \Vells; as results percolate
  down the tree, spid vectors and corresponding perturbed values from different
  solvable \Parm{}s are merged. The \Vells\ received by the solver contain the
  full set of spids from all the solvable \Parm{}s in its trees.

  \item The Solver computes a set of incremental updates. Note that the Solver
  knows nothing of \Polc{}s or \Parm{}s; instead, it only deals with abstract
  ``atomic parameters'' identified by spid.

  \item 
  
  \end{itemize}
  
  

  Presumably, at some point the new polc values need to be stored into a MEP
  table. There are two ways to accomplish this:

  \begin{description}
  
  \item[Automatically:] if the \qq{auto\_save} flag is set in the \Parm\ state
  record, then each update via \qq{set\_value} is immediately committed to the
  MEP table.

  \item[With explicit command:] whenever the \Parm\ receives a request with a
  \qq{save\_polc=T} command in the rider, it commits all its polcs to the MEP
  table.

  \end{description}
  
  The \qq{dbid\_index} attribute of the \Polc\ is used to keep track of its
  location in the MEP table. When a polc is loaded from the table (whether on
  the C++ side, or via the Glish \qq{meptable()} object), \qq{dbid\_index} is
  set to its DB identifier.\footnote{As long as we use AIPS++ tables, this is
  simply the row number. If and when we employ other storage schemes, dbid 
  may become something more complex.}  When a polc is subsequently saved, its
  dbid is used to locate the correct entry in the table. New polcs are created
  with a dbid of -1; when they are subsequently saved, a new entry is allocated
  in the MEP table, and the polc object is updated with the dbid. During normal
  operation, the user need not worry about this since everything happens
  automatically; some advanced scripting may require knowledge of the dbid.

\section{MeqServer interface}

Coming soon.

\end{document}
